\documentclass[a4paper]{scrartcl}

\usepackage[
    fancytheorems, 
    fancyproofs, 
    noindent, 
]{adam}
\usepackage{floatrow}
\usepackage{import}
\usepackage{xifthen}
\usepackage{pdfpages}
\usepackage{transparent}
\usepackage{bm}

\newcommand{\incfig}[2]{%
    \def\svgwidth{#1mm}
    \import{./figures/}{#2.pdf_tex}
}

\title{IB Groups, Rings and Modules}
\author{Martin von Hodenberg (\texttt{mjv43@cam.ac.uk})}
\date{\today}
\setcounter{section}{-1}

\allowdisplaybreaks

\begin{document}

\maketitle

These are my notes for the IB course `Groups, Rings and Modules', which was lectured in Lent 2022 at Cambridge by Dr R.Zhou. These notes are written in \LaTeX  \ for my own revision purposes. Any suggestions or feedback is welcome.

\tableofcontents
\newpage
\section{Introduction}
This course will contain several sections: 
\begin{enumerate}
    \item Groups; this will be a continuation from IA, focusing on simple groups, $p$-groups, and $p$-subgroups. The main result in this part of the course will be the Sylow theorems.
    \item Rings; these are sets where you can add, subtract and multiply (e.g $\mathbb{Z}$ or $\mathbb{C}[X]$). We will study rings of integers such as $\mathbb{Z}[i], \mathbb{Z}[\sqrt{2}]$. These also generalise to polynomial rings. We will also study fields, which are rings where you can divide (e.g $\mathbb{Q},\mathbb{R},\mathbb{C}$ or $\mathbb{Z}/p\mathbb{Z}$ for $p$ prime).
    \item Modules; these are an analogue of vector spaces where the scalars belong to a ring instead of a field. We will classify modules over certain "nice" rings. This allows us to prove Jordan Normal Form, and classify finite abelian groups. 
\end{enumerate}

\section{Groups}
\subsection{Recall of IA Groups}
\begin{definition}[Group]
     A group is a pair $(G, \cdot)$ where $G$ is a set and $\cdot: G \times G \rightarrow  G$ is a binary operation satisfying:
     \begin{enumerate}
         \item $a \cdot (b \cdot c)=(a \cdot b)\cdot c$ (associativity)
         \item $\exists e \in G$ such that $e \cdot g= g \cdot e =g$ for all $g \in G$ (identity)
         \item $\forall g \in G$, $\exists {g}^{-1} \in G$ such that $g \cdot {g}^{-1}={g}^{-1}\cdot g=e$ (inverses)
     \end{enumerate}
\end{definition}
\begin{remark}
     \begin{itemize}
         \item In practice, one often needs to check closure in order to check that $\cdot $ is well-defined.
         \item If using additive (respectively multiplicative) relations, we will often write 0 (or 1) for the identity.
         \item We write $|G|$ for the number of elements in $G$.
     \end{itemize}
\end{remark}
\begin{definition}[Subgroup]
     A subset $H \subseteq G$ is a subgroup (written $H \leq G$) if $H$ is closed under $\cdot$ and $(H, \cdot)$ is a group.
\end{definition}
\begin{remark}
     A non-empty subset $H$ of $G$ is a subgroup if $a,b \in H \implies a \cdot {b}^{-1} \in H$ (see IA Groups for the proof). 
\end{remark}
\begin{example}[Examples of groups]
    Groups we have already seen include:
    \begin{itemize}
        \item Additive groups $(\mathbb{Z},+)\leq (\mathbb{Q},+)\leq (\mathbb{R},+)$.
        \item Cyclic and dihedral groups $C_{n}$ and $D_{2n}$.
        \item Abelian groups: those groups $G$ such that $a \cdot b= b \cdot a$ for all $a,b \in G$.
        \item Symmetric and alternating groups $S_{n}=$ group of all permutations of $\left\{1, \ldots ,n\right\}$ and $A_{n} \leq S_n$, the group of all even permutations.
        \item Quaternion group $Q_8 =\left\{\pm 1, \pm i, \pm j, \pm k\right\}$ where $i,j,k$ are quaternions.
        \item General and special linear groups $GL_{n}(\mathbb{R})= n \times n$ matrices on $\mathbb{R}$ with $\operatorname{det} \neq 0 $, where the group operation is matrix multiplication. This contains the subgroup $SL_{n}(\mathbb{R}) \leq GL_{n}(\mathbb{R})$, which is the subgroup of matrices with $\operatorname{det}=1 $. 
    \end{itemize}     
\end{example}
\begin{definition}[Direct product]
     The direct product of groups $G$ and $H$ is the set $G \times H$ with operation \[
     (g_1 , h_1 ) \cdot (g_2 ,h_2 )=(g_1 g_2 , h_1 h_2 )
     .\] 
\end{definition}
\begin{theorem}[Lagrange's theorem]
     Let $H \leq G$. Then the left cosets of $H$ in $G$ are the sets $gH=\left\{gh: \ h \in H\right\}$ for $g \in G$. These partition $G$, and each has the same cardinality as $H$. From this we can deduce Lagrange's theorem:

     If $G$ is a finite group and $H \leq G$, then $|G|=|H| [G:H]$ where $[G:H]$ is the number of left cosets of $H$ in $G$ (the index of $H$ in $G$).
\end{theorem}
\begin{remark}
     Can also carry this out with right cosets. A corollary of Lagrange's theorem is thus that the number of left cosets = number of right cosets.
\end{remark}
\begin{definition}[Order of an element]
     Let $g \in G$. If $\exists n \geq 1$ such that $g^{n}=1$, then the least such $n$ is the order of $g$ in $G$. If no such $n$ exists, $g$ has infinite order.
\end{definition}
\begin{remark}
     If $g$ has order $d$, then 
     \begin{itemize}
         \item $g^{n}=1 \implies d|n$. 
         \item $\left\{1,g, \ldots , g^{d-1}\right\}\leq G$ and so if $G$ is finite, then $d | |G|$ (Lagrange).
     \end{itemize}
\end{remark}
\begin{definition}[Normal subgroup]
     A subgroup $H \leq G$ is normal if ${g}^{-1}Hg=H$ for all $g \in G$. We write $H \unlhd G$.
\end{definition}
\begin{proposition}
     If $H \unlhd G$ then the set $G/H$ of left cosets of $H$ in $G$ is a group (called the quotient group) with operation \[
     g_1 H \cdot g_2 H= g_1 g_2 H
     .\] 
\end{proposition}
\begin{proof}
     Check $\cdot $ is well-defined:

     Suppose $g_1 H=g_1 ' H$ and $g_2 H= g_2 ' H$ for some $g_1 , g_1 ',g_2 , g_2' \in G$. Then $g_1 ' =g_1 h_1 $ and $g_2 '=g_2 h_2 $ for some $h_1 , h_2 \in H$. Therefore 
     \begin{align*}
         g_1 ' g_2 ' &=g_1 h_1 g_2 h_2 \\
         &=g_1 g_2 \underbrace{({g_2 }^{-1}h_1 g_2 )}_{\in H} \underbrace{ h_2}_{\in H} 
     \end{align*}  
     Therefore $g_1 ' g_2 ' H=g_1 g_2 H$.
     Associativity is inherited from $G$, the identity is $H=eH$, and the inverse of $gH$ is ${g}^{-1}H$.
\end{proof}

\begin{definition}[Homomorphism]
     If $G,H$ are groups, then a function $\phi: G \rightarrow H$ is a group homomorphism if $\phi (g_1 g_2)=\phi (g_1 g_2 )=\phi (g_1 )\phi (g_2 )$. It has kernel \[
     \operatorname{ker} \phi= \left\{g \in G: \ \phi (g)=e\right\} \leq G
     .\]  and image \[
     \operatorname{Im} \phi = \left\{\phi (g): \ g \in G \right\} \leq H
     .\] 
\end{definition}
\begin{remark}
     If $a \in \operatorname{ker}\phi$ and $g \in G$, then 
     \begin{align*}
         \phi ({g}^{-1}ag)&=\phi ({g}^{-1}) \phi (a) \phi (g)\\
         &=\phi ({g}^{-1}) \phi (g)\\
         &= \phi ({g}^{-1}g)=\phi (e)=e.
     \end{align*}
     So ${g}^{-1}ag \in \operatorname{ker}\phi$ and hence $\operatorname{ker}\phi$ is a normal subgroup of $G$.
\end{remark}

\begin{definition}[Isomorphism]
      An isomorphism of groups is a group homomorphism that is also a bijection. We say $G$ and $H$ are isomorphic and write $G \cong H$ if there exists an isomorphism $\phi: G \rightarrow H$. (Note it follows from the definition that ${\phi}^{-1}$ is also a group homomorphism)
\end{definition}

\begin{theorem}[First Isomorphism Theorem]
      Let $\phi: G \rightarrow H$ be a group homomorphism. Then $\operatorname{ker}\phi \unlhd G$ and \[
      G/\operatorname{ker}\phi \cong \operatorname{Im} \phi
      .\] 
\end{theorem}
\begin{proof}
      Let $K=\operatorname{ker} \phi$. We have already checked $K$ is normal. Now we define $\Phi: G/K \rightarrow \operatorname{Im} \phi$ by \[
      gK \rightarrow \phi (g).
      .\]
      To show $\Phi$is well defined and injective: 
      \begin{align*}
           g_1 K= g_2 K &\iff {g_2 }^{-1}g_1 \in K\\
           &\iff \phi ({g_2 }^{-1}g_1 )=e\\
           &\iff \phi (g_1 )=\phi (g_2 ). 
      \end{align*}
      To show $\Phi$ is a group hom.: 
      \begin{align*}
           \Phi (g_1 K g_2 K)&=\Phi (g_1 g_2 K)\\
           &=\phi (g_1 g_2 )=\phi (g_1 ) \phi (g_2 )\\
           &=\Phi (g_1 K) \Phi (g_2 K)
      \end{align*}
      Showing $\Phi$ is surjective:

      Let $x \in \operatorname{Im} \phi$, say $x= \phi (g)$ for some $g \in G$. Then $x=\phi (gR)$.
\end{proof}
\begin{example}
      Let $\phi: \mathbb{C} \rightarrow \mathbb{C}^x= \left\{x \in C: x \neq 0\right\}$ given by $z \mapsto e^{z}$.

      Since $e^{z+w}=e^{z}e^{w}$, this is a group homomorphism from $(\mathbb{C},+) \rightarrow (\mathbb{C}^{x},\times)$. We have that 
      \begin{align*}
          \operatorname{ker} \phi= \left\{z \in \mathbb{C}: \ e^x=1\right\}=2 \pi i \mathbb{Z}\\
          \operatorname{Im} \phi=\mathbb{C}^{x} \ \text{by existence of log}
      \end{align*}
      Hence $\mathbb{C}/2 \pi i \mathbb{Z} \cong \mathbb{C}^{x}$. 
\end{example}

\begin{theorem}[Second Isomorphism Theorem]
      Let $H \leq G$, and $K \unlhd G$. Then $HK = \left\{hk: h \in H, k \in K\right\} \leq G$ and $H \cap K \unlhd H$. Moreover, \[
      HK/K \cong H/(H \cap K)
      .\] 
\end{theorem}
\begin{proof}
      Let $h_1 k_1 , h_2 k_2 \in HK$ (so $h_1 h_2 \in H, \ k_1 k_2 \in K$). Now \[
      h_1 k_1 {(h_2 k_2 )}^{-1}=\underbrace{h_1 {h_2 }^{-1}}_{\in H} (\underbrace{h_2 k_1 {k_2 }^{-1} {h_2 }^{-1}}_{\in K} )\in HK 
      .\] 
     Thus $HK \leq G$ (by our previous remark). Let $\phi: H \rightarrow G/K$ be given by $h \rightarrow hK$. This is the composite of $H \rightarrow G$ and the quotient map $G \rightarrow G/K$; hence $\phi$ is a group homomorphism. Thus
     \begin{align*}
          \operatorname{ker} \phi = \left\{h \in H: hK=K\right\}=H \cap K \unlhd H \\
          \operatorname{Im} \phi = \left\{hK: \ h \in H\right\}=HK/K
     \end{align*}
     Now by the First Isomorphism Theorem \[
     HK/K \cong H/(H \cap K)
     .\] 
\end{proof}
\begin{remark}
      Suppose $K \unlhd G$. There is a bijection \[
      \left\{\text{subgroups of }G/K\right\} \leftrightarrow \left\{\text{subgroups of }G \text{ containing }K\right\}
      ,\] 
     where $X \mapsto \left\{g \in G: gK \in X\right\}$ and $H/K \leftarrow H$. This further restricts to a bijection 
     \[
      \left\{\text{normal subgroups of }G/K\right\} \leftrightarrow \left\{\text{normal subgroups of }G \text{ containing }K\right\}
      ,\] 
\end{remark}
\begin{theorem}[Third Isomorphism Theorem]
      Let $K \leq H \leq G$ be normal subgroups of $G$. Then \[
      \frac{G/K}{H/K} \cong G/H
      .\] 
\end{theorem}
\begin{proof}
      Let $\phi: G/K \rightarrow G/K$ be defined by $gK \mapsto gH$. If $g_1 K= g_2 K$, then ${g_2 }^{-1}g_1 \in K \leq H \implies  g_1 H= g_2 H$. Thus $\phi$ is well-defined.
      
      Thus $\phi$ is a surjective homomorphim with kernel $H/K$. Now just apply the First Isomorphism Theorem.
\end{proof}
\subsection{Simple groups}
If $K \unlhd G$, then studying the groups $K$ and $G/K$ gives some information about $G$. However, this approach is not always available. This is the case when a group is simple. 
\begin{definition}[Simple group]
      A group $G$ is simple if $\left\{e\right\}$ and $G$ are its only normal subgroups.
\end{definition}
\begin{remark}
     It is convention to not consider the trivial group a simple group.
\end{remark}
\begin{lemma}
     Let $G$ be an abelian group. $G$ is simple iff $G \cong C_{p}$ for some prime $p$. 
\end{lemma}
\begin{proof}
      \textbf{$\impliedby$:}
      Let $H \leq C_{p}$. Lagrange's theorem says that $|H| \ | |C_{p}|=p$. Since $p$ is prime, $|H|=1$ or $p$. So $H$ is the trivial group or $C_{p}$.  

      \textbf{$\implies $:}
      Let $g \in G$ where $g \neq e$. Consider the subgroup generated by $g$: \[
      \langle g \rangle = \left\{\ldots , g^{-2},g^{-1},e, g, g^2, \ldots \right\}
      .\]  This is normal in $G$ since $G$ is abelian. Since $G$ is simple, $\langle g \rangle =G$. If $G$ is infinite, $G \cong (\mathbb{Z}, +)$ and $2\mathbb{Z} \leq \mathbb{Z}$ which gives a contradiction. 

      Otherwise, we now know $G \cong C_{n}$ for some $n$. Let $g$ be a generator. If $m |n $ then $g^{n/m}$ generates a subgroup of order $m$ and so $G$ simple $\implies $ the only factors of $n$ are 1 and $n$. Therefore $n$ is prime.  
\end{proof}
\begin{lemma}
     If $G$ is a finite group, then $G$ has a composition series \[
     e =G_{0} \unlhd G_1 \unlhd \ldots \unlhd G_{m}=G
     ,\] with each quotient $G_{i}/G_{i-1}$ simple.
\end{lemma}
\end{document}