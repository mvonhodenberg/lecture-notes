\documentclass[a4paper]{scrartcl}

\usepackage[
    fancytheorems, 
    fancyproofs, 
    noindent, 
]{adam}
\usepackage{floatrow}
\usepackage{import}
\usepackage{xifthen}
\usepackage{pdfpages}
\usepackage{transparent}

\newcommand{\incfig}[2]{%
    \def\svgwidth{#1mm}
    \import{./figures/}{#2.pdf_tex}
}

\title{IB Groups, Rings and Modules}
\author{Martin von Hodenberg (\texttt{mjv43@cam.ac.uk})}
\date{Last updated: \today}
\setcounter{section}{-1}

\allowdisplaybreaks

\begin{document}

\maketitle
These are my notes for the IB course `Groups, Rings and Modules', which was lectured in Lent 2022 at Cambridge by Dr R.Zhou. These notes are written in \LaTeX  \ for revision purposes\footnote{These notes are posted online on \href{https://mjv43.user.srcf.net/}{my website}.}. Any suggestions or feedback is welcome.
\tableofcontents

\newpage
\section{Introduction}
This course will contain several sections: 
\begin{enumerate}
    \item Groups; this will be a continuation from IA, focusing on simple groups, $p$-groups, and $p$-subgroups. The main result in this part of the course will be the Sylow theorems.
    \item Rings; these are sets where you can add, subtract and multiply (e.g $\mathbb{Z}$ or $\mathbb{C}[X]$). We will study rings of integers such as $\mathbb{Z}[i], \mathbb{Z}[\sqrt{2}]$. These also generalise to polynomial rings. We will also study fields, which are rings where you can divide (e.g $\mathbb{Q},\mathbb{R},\mathbb{C}$ or $\mathbb{Z}/p\mathbb{Z}$ for $p$ prime).
    \item Modules; these are an analogue of vector spaces where the scalars belong to a ring instead of a field. We will classify modules over certain "nice" rings. This allows us to prove Jordan Normal Form, and classify finite abelian groups. 
\end{enumerate}
\section{Groups}
\subsection{Recall of IA Groups}
This first subsection will just recap the results seen in IA Groups; it can be skipped by anyone with a sufficient knowledge of the course.
\begin{definition*}[Group]
     A group is a pair $(G, \cdot)$ where $G$ is a set and $\cdot: G \times G \rightarrow  G$ is a binary operation satisfying:
     \begin{enumerate}
         \item $a \cdot (b \cdot c)=(a \cdot b)\cdot c$ (associativity)
         \item $\exists e \in G$ such that $e \cdot g= g \cdot e =g$ for all $g \in G$ (identity)
         \item $\forall g \in G$, $\exists {g}^{-1} \in G$ such that $g \cdot {g}^{-1}={g}^{-1}\cdot g=e$ (inverses)
     \end{enumerate}
\end{definition*}
\begin{remarks}\hfill
     \begin{itemize}
         \item In practice, one often needs to check closure in order to check that $\cdot $ is well-defined.
         \item If using additive (respectively multiplicative) relations, we will often write 0 (or 1) for the identity.
         \item We write $|G|$ for the number of elements in $G$.
     \end{itemize}
\end{remarks}
\begin{definition*}[Subgroup]
     A subset $H \subseteq G$ is a subgroup (written $H \leq G$) if $H$ is closed under $\cdot$ and $(H, \cdot)$ is a group.
\end{definition*}
\begin{remark}
     A non-empty subset $H$ of $G$ is a subgroup if $a,b \in H \implies a \cdot {b}^{-1} \in H$ (see IA Groups for the proof). 
\end{remark}
\begin{example*}[Examples of groups]
    Groups we have already seen include:
    \begin{itemize}
        \item Additive groups $(\mathbb{Z},+)\leq (\mathbb{Q},+)\leq (\mathbb{R},+)$.
        \item Cyclic and dihedral groups $C_{n}$ and $D_{2n}$.
        \item Abelian groups: those groups $G$ such that $a \cdot b= b \cdot a$ for all $a,b \in G$.
        \item Symmetric and alternating groups $S_{n}=$ group of all permutations of $\left\{1, \ldots ,n\right\}$ and $A_{n} \leq S_n$, the group of all even permutations.
        \item Quaternion group $Q_8 =\left\{\pm 1, \pm i, \pm j, \pm k\right\}$ where $i,j,k$ are quaternions.
        \item General and special linear groups $GL_{n}(\mathbb{R})= n \times n$ matrices on $\mathbb{R}$ with $\operatorname{det} \neq 0 $, where the group operation is matrix multiplication. This contains the subgroup $SL_{n}(\mathbb{R}) \leq GL_{n}(\mathbb{R})$, which is the subgroup of matrices with $\operatorname{det}=1 $. 
    \end{itemize}     
\end{example*}
\begin{definition*}[Direct product]
     The direct product of groups $G$ and $H$ is the set $G \times H$ with operation \[
     (g_1 , h_1 ) \cdot (g_2 ,h_2 )=(g_1 g_2 , h_1 h_2 )
     .\] 
\end{definition*}
\begin{theorem}[Lagrange's theorem]
     Let $H \leq G$. Then the left cosets of $H$ in $G$ are the sets $gH=\left\{gh: \ h \in H\right\}$ for $g \in G$. These partition $G$, and each has the same cardinality as $H$. From this we can deduce Lagrange's theorem:

     If $G$ is a finite group and $H \leq G$, then $|G|=|H| [G:H]$ where $[G:H]$ is the number of left cosets of $H$ in $G$ (the index of $H$ in $G$).
\end{theorem}
\begin{remark}
     Can also carry this out with right cosets. A corollary of Lagrange's theorem is thus that the number of left cosets = number of right cosets.
\end{remark}
\begin{definition*}[Order of an element]
     Let $g \in G$. If $\exists n \geq 1$ such that $g^{n}=1$, then the least such $n$ is the order of $g$ in $G$. If no such $n$ exists, $g$ has infinite order.
\end{definition*}
\begin{remark}
     If $g$ has order $d$, then 
     \begin{itemize}
         \item $g^{n}=1 \implies d|n$. 
         \item $\left\{1,g, \ldots , g^{d-1}\right\}\leq G$ and so if $G$ is finite, then $d | |G|$ (Lagrange).
     \end{itemize}
\end{remark}
\begin{definition*}[Normal subgroup]
     A subgroup $H \leq G$ is normal if ${g}^{-1}Hg=H$ for all $g \in G$. We write $H \unlhd G$.
\end{definition*}
\begin{proposition}
     If $H \unlhd G$ then the set $G/H$ of left cosets of $H$ in $G$ is a group (called the quotient group) with operation \[
     g_1 H \cdot g_2 H= g_1 g_2 H
     .\] 
\end{proposition}
\begin{proof}
     Check $\cdot $ is well-defined:

     Suppose $g_1 H=g_1 ' H$ and $g_2 H= g_2 ' H$ for some $g_1 , g_1 ',g_2 , g_2' \in G$. Then $g_1 ' =g_1 h_1 $ and $g_2 '=g_2 h_2 $ for some $h_1 , h_2 \in H$. Therefore 
     \begin{align*}
         g_1 ' g_2 ' &=g_1 h_1 g_2 h_2 \\
         &=g_1 g_2 \underbrace{({g_2 }^{-1}h_1 g_2 )}_{\in H} \underbrace{ h_2}_{\in H} 
     \end{align*}  
     Therefore $g_1 ' g_2 ' H=g_1 g_2 H$.
     Associativity is inherited from $G$, the identity is $H=eH$, and the inverse of $gH$ is ${g}^{-1}H$.
\end{proof}

\begin{definition*}[Homomorphism]
     If $G,H$ are groups, then a function $\phi: G \rightarrow H$ is a group homomorphism if $\phi (g_1 g_2)=\phi (g_1 g_2 )=\phi (g_1 )\phi (g_2 )$. It has kernel \[
     \operatorname{ker} \phi= \left\{g \in G: \ \phi (g)=e\right\} \leq G
     .\]  and image \[
     \operatorname{Im} \phi = \left\{\phi (g): \ g \in G \right\} \leq H
     .\] 
\end{definition*}
\begin{remark}
     If $a \in \operatorname{ker}\phi$ and $g \in G$, then 
     \begin{align*}
         \phi ({g}^{-1}ag)&=\phi ({g}^{-1}) \phi (a) \phi (g)\\
         &=\phi ({g}^{-1}) \phi (g)\\
         &= \phi ({g}^{-1}g)=\phi (e)=e.
     \end{align*}
     So ${g}^{-1}ag \in \operatorname{ker}\phi$ and hence $\operatorname{ker}\phi$ is a normal subgroup of $G$.
\end{remark}

\begin{definition*}[Isomorphism]
      An isomorphism of groups is a group homomorphism that is also a bijection. We say $G$ and $H$ are isomorphic and write $G \cong H$ if there exists an isomorphism $\phi: G \rightarrow H$. (Note it follows from the definition that ${\phi}^{-1}$ is also a group homomorphism)
\end{definition*}

\begin{theorem}[First Isomorphism Theorem]
      Let $\phi: G \rightarrow H$ be a group homomorphism. Then $\operatorname{ker}\phi \unlhd G$ and \[
      G/\operatorname{ker}\phi \cong \operatorname{Im} \phi
      .\] 
\end{theorem}
\begin{proof}
      Let $K=\operatorname{ker} \phi$. We have already checked $K$ is normal. Now we define $\Phi: G/K \rightarrow \operatorname{Im} \phi$ by \[
      gK \rightarrow \phi (g).
      .\]
      To show $\Phi$ is well defined and injective: 
      \begin{align*}
           g_1 K= g_2 K &\iff {g_2 }^{-1}g_1 \in K\\
           &\iff \phi ({g_2 }^{-1}g_1 )=e\\
           &\iff \phi (g_1 )=\phi (g_2 ). 
      \end{align*}
      To show $\Phi$ is a group hom.: 
      \begin{align*}
           \Phi (g_1 K g_2 K)&=\Phi (g_1 g_2 K)\\
           &=\phi (g_1 g_2 )=\phi (g_1 ) \phi (g_2 )\\
           &=\Phi (g_1 K) \Phi (g_2 K)
      \end{align*}
      Showing $\Phi$ is surjective:

      Let $x \in \operatorname{Im} \phi$, say $x= \phi (g)$ for some $g \in G$. Then $x=\phi (gR)$.
\end{proof}
\begin{example*}
      Let $\phi: \mathbb{C} \rightarrow \mathbb{C}^x= \left\{x \in C: x \neq 0\right\}$ given by $z \mapsto e^{z}$.

      Since $e^{z+w}=e^{z}e^{w}$, this is a group homomorphism from $(\mathbb{C},+) \rightarrow (\mathbb{C}^{x},\times)$. We have that 
      \begin{align*}
          \operatorname{ker} \phi= \left\{z \in \mathbb{C}: \ e^x=1\right\}=2 \pi i \mathbb{Z}\\
          \operatorname{Im} \phi=\mathbb{C}^{x} \ \text{by existence of log}
      \end{align*}
      Hence $\mathbb{C}/2 \pi i \mathbb{Z} \cong \mathbb{C}^{x}$. 
\end{example*}

\begin{theorem}[Second Isomorphism Theorem]
      Let $H \leq G$, and $K \unlhd G$. Then $HK = \left\{hk: h \in H, k \in K\right\} \leq G$ and $H \cap K \unlhd H$. Moreover, \[
      HK/K \cong H/(H \cap K)
      .\] 
\end{theorem}
\begin{proof}
      Let $h_1 k_1 , h_2 k_2 \in HK$ (so $h_1 h_2 \in H, \ k_1 k_2 \in K$). Now \[
      h_1 k_1 {(h_2 k_2 )}^{-1}=\underbrace{h_1 {h_2 }^{-1}}_{\in H} (\underbrace{h_2 k_1 {k_2 }^{-1} {h_2 }^{-1}}_{\in K} )\in HK 
      .\] 
     Thus $HK \leq G$ (by our previous remark). Let $\phi: H \rightarrow G/K$ be given by $h \rightarrow hK$. This is the composite of $H \rightarrow G$ and the quotient map $G \rightarrow G/K$; hence $\phi$ is a group homomorphism. Thus
     \begin{align*}
          \operatorname{ker} \phi = \left\{h \in H: hK=K\right\}=H \cap K \unlhd H \\
          \operatorname{Im} \phi = \left\{hK: \ h \in H\right\}=HK/K
     \end{align*}
     Now by the First Isomorphism Theorem \[
     HK/K \cong H/(H \cap K)
     .\] 
\end{proof}
\begin{remark}[1.2]
      Suppose $K \unlhd G$. There is a bijection \[
      \left\{\text{subgroups of }G/K\right\} \leftrightarrow \left\{\text{subgroups of }G \text{ containing }K\right\}
      ,\] 
     where $X \mapsto \left\{g \in G: gK \in X\right\}$ and $H/K \leftarrow H$. This further restricts to a bijection 
     \[
      \left\{\text{normal subgroups of }G/K\right\} \leftrightarrow \left\{\text{normal subgroups of }G \text{ containing }K\right\}
      ,\] 
\end{remark}
\begin{theorem}[Third Isomorphism Theorem]
      Let $K \leq H \leq G$ be normal subgroups of $G$. Then \[
      \frac{G/K}{H/K} \cong G/H
      .\] 
\end{theorem}
\begin{proof}
      Let $\phi: G/K \rightarrow G/K$ be defined by $gK \mapsto gH$. If $g_1 K= g_2 K$, then ${g_2 }^{-1}g_1 \in K \leq H \implies  g_1 H= g_2 H$. Thus $\phi$ is well-defined.
      
      Thus $\phi$ is a surjective homomorphim with kernel $H/K$. Now just apply the First Isomorphism Theorem.
\end{proof}
\subsection{Simple groups}
If $K \unlhd G$, then studying the groups $K$ and $G/K$ gives some information about $G$. However, this approach is not always available. This is the case when a group is simple. 
\begin{definition*}[Simple group]
      A group $G$ is simple if $\left\{e\right\}$ and $G$ are its only normal subgroups.
\end{definition*}
\begin{remark}
     It is convention to not consider the trivial group a simple group.
\end{remark}
\begin{lemma}\label{abeliansimpleiffCp}
     Let $G$ be an abelian group. $G$ is simple iff $G \cong C_{p}$ for some prime $p$. 
\end{lemma}
\begin{proof}
      \textbf{$\impliedby$:}
      Let $H \leq C_{p}$. Lagrange's theorem says that $|H| \ | |C_{p}|=p$. Since $p$ is prime, $|H|=1$ or $p$. So $H$ is the trivial group or $C_{p}$.  

      \textbf{$\implies $:}
      Let $g \in G$ where $g \neq e$. Consider the subgroup generated by $g$: \[
      \langle g \rangle = \left\{\ldots , g^{-2},g^{-1},e, g, g^2, \ldots \right\}
      .\]  This is normal in $G$ since $G$ is abelian. Since $G$ is simple, $\langle g \rangle =G$. If $G$ is infinite, $G \cong (\mathbb{Z}, +)$ and $2\mathbb{Z} \leq \mathbb{Z}$ which gives a contradiction. 

      Otherwise, we now know $G \cong C_{n}$ for some $n$. Let $g$ be a generator. If $m |n $ then $g^{n/m}$ generates a subgroup of order $m$ and so $G$ simple $\implies $ the only factors of $n$ are 1 and $n$. Therefore $n$ is prime.  
\end{proof}
\begin{lemma}\label{compositionserieslemma}
     If $G$ is a finite group, then $G$ has a composition series \[
     e =G_{0} \unlhd G_1 \unlhd \ldots \unlhd G_{m}=G
     ,\] with each quotient $G_{i}/G_{i-1}$ simple.
\end{lemma}
\begin{proof}
      We induct on $|G|$. If $|G|=1$ it's obvious. If $|G|>1$, let $G_{m-1}$ be a normal subgroup of largest possible order that isn't $G$ itself. Remark 1.2 implies $G/G_{m-1}$ is simple. Then apply the induction hypothesis to $G_{m-1}$. 
\end{proof}
\subsection{Group actions}
\begin{definition*}[Permutation group]
      For $X$ any set, let $\operatorname{Sym}(X)$ be the group of all bijections $X \rightarrow X$ under composition. This clearly forms a group with $e=\operatorname{Id}_X$.

      A group $G$ is a permutation group of degree n if $G \leq \operatorname{Sym}(X)$ with $|X|=n$. 
\end{definition*}
\begin{example*}[Examples of permutation group]\hfill
      \begin{itemize}
           \item $S_{n}=\operatorname{Sym}(\left\{1,2,\ldots ,n\right\})$ is a permutation group of degree $n$, as is $A_{n}\leq S_{n}$.
           \item $D_{2n}=$ (symmetries of a regular $n$-gon) is a subgroup of Sym($\left\{\text{vertices of n-gon}\right\}$).
      \end{itemize}
\end{example*}
\begin{definition*}[Group action]
      An actio of a group $G$ on a set $X$ is a function $\ast: G \times X \rightarrow X$ satisfying 
      \begin{itemize}
           \item[(i)] $e \ast x =x $ for all $x \in X$ 
           \item[(ii)] $(g_1 g_2 )\ast x= g_1 \ast (g_2 \ast x)$ for all $g_1 ,g_2 \in G$, $x \in X$.
      \end{itemize}
\end{definition*}
\begin{proposition}
     An action of a group $G$ on a set $X$ is equivalent to specifying a group homomorphism $\phi: G \rightarrow \operatorname{Sym}(X)$.
\end{proposition}
\begin{proof}
      For each $g \in G$, let $\phi_{g}: X \rightarrow X$ send $x \mapsto g \ast x$. 

      We have $\phi_{g_1 g_2 }(x)=(g_1 g_2 )\ast x=g_1 \ast (g_2 \ast x)=\phi_{g_1 }\circ \phi_{g_2 }(x)$. ($\dagger$)

      In particular, $\phi_{g}\circ \phi_{{g}^{-1}}=\phi_{{g}^{-1}}\circ \phi_{g}=\phi_{e}=\operatorname{Id}_X$. Thus $\phi_{g} \in \operatorname{Sym}(X)$. Then the map $\phi: G \rightarrow \operatorname{Sym}(X)$ given by $g \mapsto \phi_{g}$ is a group homomorphism by ($\dagger$).

      Conversely, let $\phi: G \rightarrow \operatorname{Sym}(X)$ be a group homomorphism. Define $\ast: G \times X \rightarrow X$ given by $(g,x) \mapsto \phi (g)(x)$. Then
      \begin{itemize}
           \item[(i)] $e \ast x=\phi (e)(x)=\operatorname{Id}_X (x)=x$.
           \item[(ii)] $(g_1 g_2 )\ast x= \phi (g_1 g_2 )(x)=\phi (g_1 )(\phi (g_2)(x))=g_1 \ast (g_2 \ast x)$. 
      \end{itemize}
\end{proof}
\begin{definition*}
      We say $\phi: G \rightarrow \operatorname{Sym}(X)$ is a permutation representation of $G$.
\end{definition*}
\begin{definition*}[Orbit and stabiliser]
     Let $G$ act on a set $X$. 
     \begin{itemize}
          \item[(i)] The orbit of $x \in X$ is $\operatorname{orb}_G (x)=\left\{g \ast x: \ g \in G\right\} \subset X$
          \item[(ii)] The stabiliser of $x \in X$ is \[
          G_{x}=\left\{g \in G: \ g \ast x=x\right\} \leq G
          .\] 
     \end{itemize}
\end{definition*}
Recall the Orbit-Stabiliser Theorem from IA Groups: There is a bijection $\operatorname{orb}_G (x) \leftrightarrow $ the set of left cosets of $G_{x}$ in $G$. In particular if $G$ is finite, then \[
|G|=|\operatorname{orb}_G (x)| |G_{x}|
.\] 
\begin{example*}[Example of Orbit-Stabiliser]
      Let $G$ be the group of all symmetries of a cube, acting on the set of veretices $X$. We can reach any vertex from any other one, so $|\operatorname{orb}_G (x)|=8$. Some basic geometry gives $|G_{x}|=6$. Therefore $|G|=48$.
\end{example*} 
\begin{remark}
      \begin{itemize}
           \item $\operatorname{ker} \phi=\bigcap_{x \in X}G_{x}$ is called the kernel of the group action. 
           \item The orbits partition $X$. We say the action is transitive if there is only one orbit.
           \item $G_{g \ast x}=g G_{x} {g}^{-1}$, so if $x,y \in X$ belong to the same orbit, then their stabilisers are conjugate.
      \end{itemize}
\end{remark}
Later on a lot of the proofs will involve picking a nice group action. So let's look at some examples of group actions. 
\begin{itemize}
     \item[(i)] Let $G$ act on itself by left multiplication, i.e $g \ast x =gx$. The kernel of this action is \[
     \left\{g \in G: gx=x \quad \forall x \in G\right\}=e
     .\] Thus $G $ is injective into $\operatorname{Sym}(G)$. This proves Cayley's theorem:
 \end{itemize}
 \begin{theorem}[Cayley's theorem]
     Any finite group $G$ is isomorphic to a subgroup of the symmetric group $S_{n}$ for some $n$. (Take $n=|G|$.)
 \end{theorem}
 \begin{proof}
       As above in (i).
 \end{proof}
 \begin{itemize}
      \item[(ii)] Let $H \leq G$; then $G$ acts on $G/H$ by left multiplication, i.e $g \ast x H= gxH$. This action is transitive (since $(x_2 {x_1 }^{-1})x_1 H=x_2 H$) with 
      \begin{align*}
          G_{xH}&=\left\{g \in G: gxH=xH\right\}\\&=\left\{g \in G: {x}^{-1}gx \in H\right\}\\&=xH {x}^{-1} 
      \end{align*} Thus $\operatorname{ker}(\phi)=\bigcap_{x \in G}xH {x}^{-1}$. This is the largest normal subgroup of $G$ that is contained in $H$.  
 \end{itemize}
 \begin{theorem}
       Let $G$ be a non-abelian simple group, and $H \leq G$ a subgroup of index $n>1$. Then $n \geq (5$ and $G$ is isomorphic to a subgroup of $A_{n}$.
 \end{theorem}
 \begin{proof}
     Let $G$ act on $X=G/H$ by left multiplication, and let $\phi: G \rightarrow \operatorname{Sym}(X)$ be the associated permutation representation. As $G$ is simple, $\operatorname{ker}(\phi)=e$ or $G$. If $\operatorname{ker}(\phi)=G$, then $\operatorname{Im}(\phi)=e$. This is a contradiction since $G$ acts transitively on $X$ and $|X|>1$. Thus $\operatorname{ker}(\phi)=e$ and $G \cong \operatorname{Im}(\phi) \leq S_{n}.$ Since $G \leq S_{n}$ and $A_n \unlhd S_n$, the second isomorphism theorem gives $G \cap A_n \unlhd G$ and $G/(G \cap A_n) \cong GA_n/A_n \leq S_n / A_n \cong C_2 $. Since $G$ is simple, $G \cap A_n=e$ (this is impossible as $G \leq C_2 $ but $G$ isn't abelian) or $G$. Thus $G \leq A_{n} $. Finally, if $n \leq 4$, then $A_n$ has no non-abelian simple subgroups.
 \end{proof}
\begin{itemize}
     \item[(iii)] Let $G$ act on itself by conjugation, i.e $g \ast x= gx {g}^{-1}$. We define the conjugacy class of $x \in G$ to be \[
     \operatorname{ccl}_G (x)=\operatorname{orb}_{G}(x)=\left\{gx {g}^{-1} \in G: \quad g \in G\right\}
     .\] We also define the centraliser of $x$ by \[
     C_{G}(x)=G_{x}= \left\{g \in G: \quad gx=xg\right\} \leq G
     .\] We define the centre of $G$ by \[
     Z (G)=\operatorname{Ker}(\phi)=\left\{g \in G: \quad gx=xg \ \forall x \in G\right\}
     .\] Note that the $\phi (g): G \rightarrow G$ given by $h \mapsto gh {g}^{-1}$ satisfies \[
     \phi (g)(h_1 h_2 )=gh_1 h_2 {g}^{-1}=gh_1 {g}^{-1} gh_2 {g}^{-1} =\phi (g)(h_1 )\phi (g)(h_2 )
     .\]  Thus $\phi (g)$ is a group homomorphism, and also a bijection i.e $\phi (g)$ is an isomorphism. 
\end{itemize}
\begin{definition*}[Automorphism]
      $\operatorname{Aut}(G)=\left\{\text{ group isomorphisms } \zeta: G \rightarrow G\right\}$. Then $\operatorname{Aut}(G)\leq \operatorname{Sym}(G)$ and $\phi: G \rightarrow \operatorname{Sym}(G)$ has image in $\operatorname{Aut}(G).$ 
\end{definition*}
\begin{itemize}
     \item[(iv)] Let $X$ be the set of all subgroups of $G$. Then $G$ acts on $X$ by conjugation, i.e $g \ast H=gH {g}^{-1}$. The stabiliser of $H$ is \[
     \left\{g \in G: \quad g H {g}^{-1}=H\right\}=N_{G}(H)
     .\] This is also called the normaliser of $H$ in $G$, and is the largest subgroup of $G$ containing $H$ as a normal subgroup. In particular, \[
     H \unlhd G \iff N_{G}(H) =G
     .\] 
\end{itemize}
\subsection{Alternating groups}
From IA Groups, we know that elements in $S_{n}$ are conjugate iff they have the same cycle type. For example, in $S_5$, we have the following:
\begin{center}
\begin{tabular}{|c|c|c|}
     \hline
     Cycle type & Number of elements & Sign\\
     \hline
     id &1&+1\\
     ($\ast$$\ast$)&10&-1\\
     ($\ast$$\ast$)($\ast$$\ast$)&15&+1\\
     ($\ast$$\ast$$\ast$)& 20&+1\\
     ($\ast$$\ast$)($\ast$$\ast$$\ast$) & 20&-1\\
     ($\ast$$\ast$$\ast$$\ast$) & 30&-1\\
     ($\ast$$\ast$$\ast$$\ast$$\ast$) & 24&+1\\
     \hline
     Total:& 120=5!=$|S_5 |$&\\
     \hline
\end{tabular}
\end{center}


Let $g \in A_n$. Then $C_{A_n}(g)=C_{S_n}(g) \cap A_{n}$. We effectively have two cases: 
\begin{itemize}
     \item If there exists an odd permutation commuting with $g$, then $|C_{A_{n}}(g)|=\frac{1}{2}|C_{S_{n}}(g)|$ and by Orbit-Stabiliser, $|\operatorname{ccl}_{A_{n}}(g)|=|\operatorname{ccl}_{S_{n}}(g)|$.
     \item Otherwise,  $|C_{A_{n}}(g)|=|C_{S_{n}}(g)|$ and by Orbit-Stabiliser, $|\operatorname{ccl}_{A_{n}}(g)|=\frac{1}{2}|\operatorname{ccl}_{S_{n}}(g)|$.
\end{itemize}
\begin{example*}[Conjugacy classes of $A_5 $]
      If we take $n=5$, then first consider the element (12)(34), which commutes with (12). Also, (123) commutes with (45). 

      But if we take $g=(12345)$, then $h \in C_{S_5 }(g)$ means 
      \begin{align*}
           (12345)&=h (12345) {h}^{-1}\\
           &= (h (1) h (2) h (3) h (4) h (5))
           \implies h \in \langle g \rangle \leq A_5 . 
      \end{align*}
      In this case, the conjugacy class does split. 

      Thus $A_5 $ has conjugacy classes of sizes 1,15,20,12,12.
\end{example*}
\begin{proposition}
     $A_5 $ is simple.
\end{proposition}
\begin{proof}
      If $H \unlhd A_5$, then $H$ is a union of conjugacy classes. Therefore \[
      |H|=1+15a+20b+12c \quad \text{ for some } a,b \in \left\{0,1\right\} \text{ and } c \in \left\{0,1,2\right\}
      .\] Since $H |60$, this implies $H=1$ or 60, i.e $A_5 $ is simple. 
\end{proof}
Now we move on to a more general statement about $A_n$ being simple. Before we can do that, we will need some lemmas for the proof. 
\begin{lemma}\label{an1}
      $A_n$ is generated by 3-cycles. 
\end{lemma}
\begin{proof}
      Each $\sigma \in A_n$ is a product of an even number of transpositions. Thus it suffices to write the product of any two transpositions as a product of 3-cycles. For $a,b,c,d$ distinct, we can have 
      \begin{equation*}
            \begin{cases}
                 (ab)(bc)=(abc)\\
                 (ab)(cd)=(acb)(acd).
            \end{cases}  
      \end{equation*}
\end{proof}
\begin{lemma}\label{an2}
     If $n \geq 5$, then all 3-cycles in $A_n$ are conjugate. 
\end{lemma}
\begin{proof}
     We claim that every 3-cycle is conjugate to (123). Indeed, if $(abc)$ is a 3-cycle, then $(abc)=\sigma (abc) {\sigma}^{-1}$ for some $\sigma \in S_n$. If $\sigma \notin A_{n}$, then replace $\sigma$ by $\sigma (45)$ (using the fact that $n \geq 5$).
\end{proof}
\begin{theorem}
      $A_n$ is simple for all $n \geq 5$.
\end{theorem}
\begin{proof}
      Let $e \neq N \unlhd A_{n}$. Suffices to show that $N$ contains a 3-cycle, since by \ref{an1} and \ref{an2} we then have $N=A_n$.

      Take $e \neq \sigma \in N$ and write $\sigma$ in its disjoint cycle decomposition. Consider the cases: 
      \begin{enumerate}
           \item  $\sigma$ contains a cycle of length $r \geq 4$. WLOG $\sigma=(123 \ldots r)\tau$, where $\tau$ is some product of cycles that fixes $1,2, \ldots ,r$. 
           
           Let $\delta =(123)$. Then consider the element \[
           \underbrace{{\sigma}^{-1}}_{\in N} \underbrace{{\delta}^{-1} \sigma \delta}_{\in N} = (r \ldots 21)(132)(12 \ldots r)(123)=(23r) \in N
           .\] Note $\tau$ gets cancelled as it fixes 1 to $r$. Therefore $N$ contains a 3-cycle. 
           \item $\sigma$ contains two 3-cycles. WLOG $\sigma= (123)(456)\tau$. Let $b=(124).$ Then \[
               {\sigma}^{-1} {\delta}^{-1} \sigma \delta = (132)(465)(142)(123)(456)(124)=(12436) \in N
          .\] Then we are back to Case 1, so $N$ contains a 3-cycle..
          \item $\sigma$ contains two 2-cycles. WLOG $\sigma=(12)(34)\tau$. Let $\delta=(123)$ and consider \[
               {\sigma}^{-1} {\delta}^{-1} \sigma \delta = (12)(34)(132)(12)(34)(123)=(14)(23)=\pi \in N
               .\] Let $\varepsilon= (235)$. Then \[
               {\pi}^{-1}{\varepsilon}^{-1}\pi \varepsilon =(14)(23)(253)(14)(23)(235)=(253)
               .\] Thus $N$ contains a 3-cycle.
      \end{enumerate}
      To conclude the proof, we consider the remaining cases:
      \begin{enumerate}
           \item Cycle type ($\ast$ $\ast$) $\implies \sigma \notin A_n$.
           \item Cycle type ($\ast$$\ast$$\ast$) $\implies \sigma$ is a 3-cycle.
           \item Cycle type ($\ast$$\ast$)($\ast$$\ast$$\ast$) $\implies  \sigma \notin A_n$. 
      \end{enumerate}
      This concludes the proof.
\end{proof}
\subsection{$p$-groups and $p$-subgroups}
\begin{definition*}[$p$-group]
      Let $p$ be a prime. A finite group $G$ is a $p$-group if $|G|=p^{n}$, $n \geq 1$. 
\end{definition*}
\begin{theorem}\label{pgroup1}
      If $G$ is a $p$-group, then $Z (G) \neq 1$.
\end{theorem}
\begin{proof}
      For $g \in G$, we have $|\operatorname{ccl}_{G}(g)||C_{G}(g)|=|G|=p^{n}$. So each conjugacy class must have size that is a power of $p$. Since $G$ is a dijoint union of conjugacy classes,
      \begin{align*}
          |G|\equiv (\text{number of conjugacy classes of size } 1)& \mod p\\
          \implies 0 \equiv |Z (G)| & \mod p\\
          \implies Z (G) \neq 1.
      \end{align*}
      We have used the fact that the conjugacy classes of size 1 are precisely the elements of $Z (G)$: \[
      g \in Z(G) \iff {x}^{-1}gx =g \ \forall x \in G \iff \operatorname{ccl}_{G}(g)=\left\{g\right\}
      .\] 
\end{proof}
\begin{corollary}\label{simplepgroup}
     The only simple $p$-group is $C_p$. 
\end{corollary}
\begin{proof}
      Let $G$ be a simple $p$-group. Since $Z (G) \unlhd G$, we have $Z (G)=1$ or $G$. By \ref{pgroup1}, we must have $Z (G)=G$. Therefore $G$ is abelian. Conclude by Lemma \ref{abeliansimpleiffCp}.
\end{proof}
\begin{corollary}
     Let $G$ be a $p$-group of order $p^{n}$. Then $G$ has a subgroup of order $p^{r}$ for $0 \leq r \leq n$.
\end{corollary}
\begin{proof}
      By Lemma \ref{compositionserieslemma}, $G$ has a composition series \[
      1=G_{0} \unlhd G_1 \unlhd \ldots \unlhd G_{m-1} \unlhd G _{m}=G
      .\] with each $G_{i}/G_{i-1}$ simple. Since $G$ is a $p$-group, all of the $G_{i}/G_{i-1}$ must be $p$-groups. Therefore have by Proposition \ref{simplepgroup} that $G_{i}/G_{p-1} \cong C_p$. 
      
      Thus $|G_{i}|=p^{i}$ for all $0 \leq i \leq m$ and $m=n$.
\end{proof}
\begin{lemma}\label{pgroup2}
      For $G$ a group, if $G/Z (G)$ is cyclic, then $G$ is abelian (so in fact $G/Z (G)=1$).
\end{lemma}
\begin{proof}
      Let $g Z (G)$ be a generator for $G/Z (G)$. Then each coset is of the form $g^{r} Z (G)$ for some $r \in \mathbb{Z}$. Thus \[
      G= \left\{g^{r}z : \ r \in \mathbb{Z}, z \in Z (G)\right\}
      .\] We now check that elements in this group always commute: 
      \begin{align*}
          g^{r_1 }z_1 g^{r_2 }z_2 &=g^{r_1 +r_2 }z_1 z_2 \quad \text{ since } z \in Z (G)\\
          &= g^{r_1 +r_2 } z_2 z_1 \\
          &= g^{r_2 }z_2 g^{r_1 }z_1.
      \end{align*}
     Therefore $G$ is abelian.
\end{proof}
\begin{corollary}
      If $|G|=p^{2}$ then $G$ is abelian.
\end{corollary}
\begin{proof}
      We know that $|Z (G)| \in \left\{1,p,p^2\right\}$. We can't have 1 by \ref{pgroup1}. If we have $p$, then $|G/Z (G)|=p$ and therefore is cyclic. Now applying \ref{pgroup2} we have that $G$ is abelian. If we have $p^2$ then $Z (G)=G$ so $G$ is abelian. 
\end{proof}
\subsection{The Sylow theorems}
\begin{theorem}[Sylow theorems]\label{sylow}
      Let $G$ be a finite group of order $p^{a}m$ where $p$ is a prime with $p \nmid m$. Then 
      \begin{itemize}
           \item[(i)] The set $\operatorname{Syl}_{p}(G)=\left\{P \leq G: \ |P|=p^{a}\right\}$ of Sylow $p$-subgroups is non-empty.
           \item[(ii)] All elements of $\operatorname{Syl}_{p}(G)$ are conjugate.
           \item[(iii)] We define $n_{p} =|\operatorname{Syl}_{p}(G)|$. This satisfies $n_{p} \equiv 1 \mod p$ and $n_{p}| \ |\ |G|$ (and so $n_{p}|m$ )
      \end{itemize}
\end{theorem}
\begin{proof}
      \begin{itemize}
           \item[(i)] Let $\Omega$ be the set of all \textbf{subsets} of $G$ of order $p^{a}$. We know that \[
           |\Omega|= {p^{a}m \choose p^{a}}=\left( \frac{p^{a}m}{p^{a}}\right)\left( \frac{p^{a}m -1}{p^{a}-1}\right)\ldots \left( \frac{p^{a}m- p^{a}+1}{1}\right)
           .\] For $0 \leq k \leq p^{a}$, the number $p^{a}m-k$ and $p^{a}-k$ are divisible by the same power of $p$. 

           Therefore $|\Omega|$ is coprime to $p$.\hfill $\quad (\dagger)$ 

           Let $G$ act on $\Omega$ by left-multiplication, i.e for $g \in G$ and $x \in \Omega$ we have \[
           g \ast X= \left\{gx: \ x \in X\right\} \in \Omega
           .\] For any $X \in \Omega$ we have \[
           |G_{x}|\ |\operatorname{orb}_{G}(X)|=|G|=p^{a}m
           .\] By $(\dagger)$, there exists $X$ such that $|\operatorname{orb}_{G}(X)|$ is coprime to $p$. This is because the orbits give a partition of $\Omega$, and so they can't all divide $p$. Thus
           \begin{equation}\label{sylproof1}
               p^{a} | \ |G_{x}
           \end{equation}
            On the other hand, if $g \in G$ and $x \in X$, then $g \in (g {x}^{-1})\ast X$ and hence
           \begin{align}
               &G = \bigcup_{g \in G} g \ast X=\bigcup_{y \in \operatorname{orb}_{G}(X)} Y \nonumber\\ 
               \implies & |G| \leq |\operatorname{orb}_{G}| \cdot  |X| \quad \text{ since } |Y|=|X| \nonumber\\
               \implies & |G_{x}|= \frac{|G|}{|\operatorname{orb}_{G}(X)|=|X|=p^{a}} \label{sylproof2}.
           \end{align}
           By \ref{sylproof1} and \ref{sylproof2}, $|G_{X}|=p^{a}$, i.e $G_{x} \in \operatorname{Syl}_{p}(G)$.
           \item[(ii)] We prove a stronger result. We claim that if $P \in \operatorname{Syl}_{p}(G)$ and $Q \leq G$ is a $p$-subgroup, then $Q \leq g {P}^{-1} {g}^{-1}$ for some $g \in G$.
           
           The proof is as follows: let $Q$ act on the set of left cosets (not a group!) $G/P$ by left multiplication, i.e $q \ast gP= qgP$. By Orbit-Stabiliser, each orbit has size $|Q|$, so either 1 or a multiple of $p$. Since $|G/P|=m$ by definition, and $m$ is coprime to $p$, there must exist some orbit of size 1, i.e 
           \begin{align*}
               &\exists g \in G: \quad qgP =gP \quad \forall q \in Q\\
               \implies & {g}^{-1} qg \in P \quad \forall q \in Q\\
               \implies & Q \leq gP {g}^{-1}.
           \end{align*}
           So we are done.
           \item[(iii)] Let $G$ act on $\operatorname{Syl}_{p}(G)$ by conjugation. Sylow (ii) tells us this action is transitive. Thus orbit-stabiliser implies \[
           n_{p}= |\operatorname{Syl}_{p}(G)| \ | \ |G|
           .\] Now to show that $n_p \equiv 1 \mod p$, let $P \in \operatorname{Syl}_{p}(G)$ act on $\operatorname{Syl}_{p}(G)$ by conjugation. The orbits have size dividing $|P|=p^{a}$, so either 1 or a multiple of $p$. To show $n_{p} \equiv 1 \mod p$, it suffices to show that $\left\{P\right\}$ is the unique orbit of size 1.

           If $\left\{Q\right\}$ is another orbit of size 1, then $P$ normalises $Q$, i.e $P \leq N_{G}(Q)$. Now $P,Q$ are both Sylow $p$-subgroups of $N_{G}(Q)$ since $|N_{G}(Q)| \leq p^{a}$. Thus by (ii), $P$ and $Q$ are conjugate in $N_{G}(Q)$ - but $Q \unlhd N_{G}(Q)$, thus $P=Q$. This completes the proof.
      \end{itemize}
\end{proof}
Now let's look at an application of these theorems.
\begin{corollary}
      If $n_{p}=1$, then the unique Sylow $p$-subgroup is normal. 
\end{corollary}
\begin{proof}
      Let $g \in G$ and $P \in \operatorname{Syl}_{p}(G)$. Then $gP {g}^{-1} \in \operatorname{Syl}_{p}(G)$, and so $gP {g}^{-1}=P$. Thus $P \unlhd G$.
\end{proof}
This is very useful to show groups of certain orders can't be simple.
\begin{example*}
      Let $|G|=1000=2^{3} \cdot 5^{3}$. Then $n_5 =1 \mod 5$ and $n_5 | 8$ so $n_5 =1$. Thus the unique Sylow 5-subgroup is normal and hence $G$ is not simple.
\end{example*}
\begin{example*}
      Let $|G|=132=2^2 \cdot 3 \cdot  11$. We have that $n_{11} =1 \mod 11$ and $n_{11} | 12$. So $n_{11} =1$ or 12. Suppose $G$ is simple. Then $n_{11} \neq 1$ (otherwise the Sylow 11-subgroup is normal). Hence $n_{11} =12$. 
      
      Now $n_3 \equiv 1 \mod 3$ and $n_3 | 44$. Thus $n_3 \in \left\{1,4,22\right\}$. But the case $n_3 =1$ can't occur as before. Now suppose $n_3 =4$. Then letting $G$ act on $\operatorname{Syl}_{3}(G)$ by conjugation gives a group homomorphism $\phi: G \rightarrow S_4$. But then $\operatorname{Ker}\phi \unlhd G \implies \operatorname{Ker}\phi =1$ or $G$. But $\operatorname{Ker}\phi=1$ would mean that $G$ injects into $S_4 $, which is a contradiction as $|G|=132>24=|S_4 |$, and $\operatorname{Ker}\phi=G$ would be a contradiction to Sylow (ii). 
      
      Thus $n_3 =22$ and $n_{11} =12$. Thus $G$ has $22 \cdot (3-1)=44$ elements of order 3 and $120= 12 \cdot (11-1)$ elements of order 11. But $44+120 > 132= |G|$ which is a contradiction. Hence $G$ is not simple.
\end{example*}
\subsection{Matrix groups}
Matrix groups provide a wealth of examples of finite groups, and are crucial in the classification of finite simple groups. First we will recap a few groups we've seen before in IA Groups.

For a field $F$, let $GL_{n}(F)$ be the set of $n \times n$ invertible matrices over $F$. 

This contains the subgroup $SL_{n}=\operatorname{Ker}(GL_{n}(F) \xrightarrow{\operatorname{det}} F^{\times})$. Here $F^{\times}=F \backslash \left\{0\right\}.$ 

Let $Z \unlhd GL_{n}(F)$ be the normal subgroup of scalar multiples of $I$. This is in fact the centre of $GL_{n}(F)$, but we won't prove this in the course since the proof is pretty involved.
\begin{definition*}
      We define the projective general linear group by \[
      PGL_{n}(F)=GL_{n}(F)/Z
      ,\] and the projective special linear group by \[
          PGL_{n}(F)= \frac{SL_{n}(F)}{Z \cap SL_{n}(F)} \cong \frac{Z \cdot  SL_{n}(F)}{Z} \leq PGL_{n}(F) \quad \text{ by 2nd isom. theorem}
      .\] 
\end{definition*}
\begin{example*}
     Consider $G=GL_{n}(\mathbb{Z}/p\mathbb{Z})$. A list of $n$ vectors in $(\mathbb{Z}/p\mathbb{Z})^{n}$ are the columns of some $A \in G$ iff they are linearly independent. Thus 
     \begin{align*}
          |G|&= \underbrace{(p^{n}-1)}_{\text{1st col.} }\underbrace{(p^{n}-p)}_{\text{2nd col.} } \underbrace{(p^{n}-p^2)}_{\text{3rd col.} } \ldots \underbrace{(p^{n}-p^{n-1})}_{\text{last col.} } \\
          &=p^{1+2+\ldots +n-1} (p^{n-1}-1)(p^{n}-1)\ldots (p-1)\\
          &=p^{n (n-1)/2} \prod_{i=1}^n (p^{i}-1).
     \end{align*}
     So the Sylow $p$-subgroups have size $p^{n (n-1)/2}$. Let \[
     U=\left\{\begin{pmatrix} 1&&&\\ &1&M& \\&&&\ddots & \\ &&&1 \end{pmatrix}\right\} \leq G
     \] be the set of upper triangular matrices with 1's on the diagonal. Then $U \in \operatorname{Syl}_{p}(G)$, since it has $n (n-1)/2$ entries and each can take $p$ values.
\end{example*}
Just as $PGL_{2}(\mathbb{C})$ acts on $\mathbb{C} \cup \left\{ \infty\right\}$ via Möbius transformations, $PGL_{2}(\mathbb{Z}/p\mathbb{Z})$ acts on $\mathbb{Z}/p\mathbb{Z} \cup \left\{ \infty\right\}$ via \[
\begin{pmatrix} a&b\\c&d  \end{pmatrix}\mapsto  \frac{az+b}{cz+d}
.\] Since scalar matrices act trivially, we obtain an action of $PGL_{2}(\mathbb{Z}/p\mathbb{Z})$.
\begin{lemma}\label{permrep}
      The permutation representation $PGL_{2}(\mathbb{Z}/p\mathbb{Z}) \rightarrow S_{p+1}$ is injective (in fact an isomorphism if $p=2$ or $3$).
\end{lemma}
\begin{proof}
      Suppose $\frac{az+b}{cz+d}=z$ for all $z \in \mathbb{Z}/p\mathbb{Z} \cup \left\{ \infty\right\}$. 
      \begin{itemize}
           \item Setting $z=0$ gives $b=0$.
           \item Setting $z= \infty$ gives $c=0$.
           \item Setting $z=1$ gives $a=d$.
      \end{itemize}
      So $\begin{pmatrix} a&b\\c&d \end{pmatrix}$ is a scalar matrix, hence it is trivial in $PGL_{2}(\mathbb{Z}/p\mathbb{Z}) $.
\end{proof}
\begin{lemma}
      If $p$ is an odd prime, then \[
      |PSL_{2}(\mathbb{Z}/p\mathbb{Z})|=p (p-1)(p+1)/2
      .\] 
\end{lemma}
\begin{proof}
      By Example 5.1, $|GL_2 (\mathbb{Z}/p\mathbb{Z})|=p (p^2-1)(p-1)$. The homomorphism $GL_2 (\mathbb{Z}/p\mathbb{Z}) \xrightarrow{\operatorname{det}} (\mathbb{Z}/p\mathbb{Z})^{\times}$ is surjective.
      
      Thus $|SL_2 (\mathbb{Z}/p\mathbb{Z})|=|GL_2 (\mathbb{Z}/p\mathbb{Z})|/(p-1)=p (p-1)(p+1)$. If $\begin{pmatrix} \lambda &0 \\ \lambda&0 \end{pmatrix} \in SL_2 (\mathbb{Z}/p\mathbb{Z})$, then $\lambda^2 \equiv 1 \mod p \implies \lambda \equiv \pm 1 \mod p$ (since $p$ is prime). 

      Thus $Z \cap SL_2 (\mathbb{Z}/p\mathbb{Z})=\left\{\pm I\right\}$ which are distinct since $p>2$. Therefore \[
          |PSL_2 (\mathbb{Z}/p\mathbb{Z})|=\frac{1}{2}|SL_2 (\mathbb{Z}/p\mathbb{Z})|=p (p-1)(p+1)/2
      .\] 
\end{proof}
\begin{example*}
      Let $G=PSL_2 (\mathbb{Z}/5\mathbb{Z})$. Then $|G|= \frac{4 \cdot 5 \cdot  6}{2}=60$. 
      
      Let $G$ act on $\mathbb{Z}/5\mathbb{Z} \cup \left\{ \infty\right\}$ via Mobius transformations. By Lemma \ref{permrep}, the permutation representation \[
      \phi: G \rightarrow \operatorname{Sym}(\left\{0,1,2,3,4\right\}\cup \infty) \cong S_{6} \quad  \text{ is injective.}
      .\]
      \textbf{Claim.} $\operatorname{Im} (\phi) \leq A_6 $, i.e $\psi: G \rightarrow S_6 \xrightarrow{\operatorname{sgn}} \left\{\pm 1\right\}$ is trivial. 
      
      \textbf{Proof.} Let $h \in G$ have order $2^{n}m$, $m$ odd. If $\psi (h^{m})=1$, then $\psi (h)^{m}=1 \implies \psi (h)=1$. So suffices to show $\psi (g)=1$ for all $g \in G$ with order a power of 2. But we know that every such $g$ belongs to a Sylow 2-subgroup. It then suffices to show $\psi (H)=1$, for $H$ a Sylow 2-subgroup (since $\operatorname{Ker} \psi$ is normal and all Sylow 2-subgroups are conjugate). Take \[
          H=\left\langle \begin{pmatrix} 2&0\\0&3 \end{pmatrix}(\pm I),\begin{pmatrix} 0&1\\-1&0 \end{pmatrix} (\pm I)\right\rangle
      .\] We compute that $\phi \begin{pmatrix} 2&0\\0&3 \end{pmatrix}= (14)(23)$, since it acts as $z \mapsto -z$. Also, $\phi \begin{pmatrix} 0&1\\-1&0 \end{pmatrix}= (0 \infty)(14)$, since it acts as $z \mapsto -\frac{1}{z}$. Thus $\psi (H)=1$. This proves the claim. 

      See Example Sheet 1 Q14 for a similar result: If $G \leq A_6 $ and $|
      G|=60$, then $G \cong A_5 $.
\end{example*}
\begin{remarks}
      (Not proved in this course)\hfill 
      \begin{itemize}
           \item $PSL_n (\mathbb{Z}/p\mathbb{Z})$ is a simple group for all $n \geq 2$ and $p$ a prime, e.g $(n,p)=(2,2), (2,3)$ (finite groups of Lie type). 
           \item The smallest non-abelian simple groups are $A_5 \cong PSL_2 (\mathbb{Z}/5\mathbb{Z})$ of order 60, and $PSL_2 (\mathbb{Z}/7\mathbb{Z})$ of order 168.
      \end{itemize}
\end{remarks}
\subsection{Finite abelian groups}
We now investigate finite abelian groups, which we can actually characterise very effectively.
\begin{lemma}
      If $m,n \in \mathbb{N}$ are coprime, then $C_m \times C_n \cong C_{mn}$.
\end{lemma}
\begin{proof}
      Let $g$ and $h$ be generators of $C_n$ and $C_m$. Then $(g,h)\in C_m \times C_n$ and $(g,h)^{r}=(g^{r}, h^{r})$. Hence 
      \begin{align*}
           (g,h)^{r}=1 &\iff m|r \text{ and } n|r\\
           & \iff mn|r \quad \text{ as } m,n \text{ coprime}.
      \end{align*}
      Thus $(g,h)$ has order $mn =|C_m \times C_n|$. So \[
      C_m \times C_n \cong C_{mn} \cong \langle (g,h) \rangle 
      .\]
\end{proof}
\begin{corollary}\label{finabl1}
     Let $G$ be a finite abelian group. Then \[
     G \cong C_{n_1 } \times C_{n_2 }\times \ldots \times C_{n_k}
     ,\] where $n_{i}$ are prime powers. 
\end{corollary}
\begin{proof}
      If $n=p_1^{a_1 }\ldots p_r ^{a_r}$, where $p_1 ,\ldots , p_r$ are distinct primes, then we just apply our previous lemma inductively to get \[
      C_{n}\cong C_{p_1 ^{a_1 }}\times C_{p_2 ^{a_2 }}\times \ldots \times C_{p_r ^{a_r }}
      .\] 
\end{proof}
\begin{theorem}\label{finabl2}
     Every finite abelian group $G$ is isomorphic to a product of cyclic groups.

     Note such an isomorphism is not unique.
\end{theorem}
\begin{proof}
      Immediate by applying Corollary \ref{finabl1}.
\end{proof}
\begin{theorem}\label{finabl3}
      Let $G$ be a finite abelian group. Then $G \cong C_{d_1 }\times C_{d_2 }\times \ldots C_{d_t}$, for some $d_1 | d_2 | \ldots | d_t$ (they are successively divisible).
\end{theorem}
\begin{remark}
      
\end{remark}
This almost immediately shows that finite abelian groups are pretty easy to work with. Let's use our results to compute what the abelian groups of various orders are in an example.
\begin{example*}
      \begin{itemize}
           \item[(i)] The abelian groups of order 8 are \[
               C_8 , C_2 \times C_4 , C_2 \times C_2 \times C2 
               .\] 
           \item[(ii)] For the abelian groups of order 12, using \ref{finabl2} we get that they are \[
           C_2 \times C_2 \times C_3 , C_4 \times C_3 
           .\] Using \ref{finabl3} we get \[
           C_2 \times C_6 , C_{12} 
           .\] This isn't a problem as these are pairwise isomorphic.
      \end{itemize}
\end{example*}
\begin{definition*}[Exponent of a group]
      The \vocab{exponent} of a group $G$ is the least integer $n \geq 1$ such that $g^{n}=1$ for all $g \in G$, i.e the LCM of the orders of the elements of $G$. For example, $A_4 $ has exponent 6.
\end{definition*}
\begin{corollary}
     Every finite abelian group contains an element whose order is the exponent of the group.
\end{corollary}
\begin{proof}
      If $G \cong C_{d_1 }\times C_{d_2 }\times \ldots C_{d_t}$ with $d_1 | d_2 | \ldots | d_t$, then every $g \in G$ has order dividing $d_{t}$, and $h \in C_{d_t}$ of order $d_t$, then $(1,1,\ldots ,h)\in G$ has order $d_t$. Thus $G$ has exponent $d_t$.
\end{proof}
\section{Rings}
\subsection{Definitions and examples}
\begin{definition*}[Ring]
       A \vocab{ring} is a triple $(R, +, \cdot )$ consisting of a set $R$ and two binary operations $+: R \times R \rightarrow R$, $\cdot  : R \times R \rightarrow R$ satisfying: 
       \begin{itemize}
             \item[(i)] Addition: $(R,+)$ is an abelian group, with identity element 0.
             \item[(ii)] Multiplication: the operation $\cdot $ is associative, and has an identity 1.
             \item[(iii)] Distributivity: $x \cdot  (y+z)= x \cdot y + x \cdot  z$ and $(y +z) \cdot  x=y \cdot x +z \cdot  x $ for all $x,y,z \in R$.
       \end{itemize}
       We say $R$ is a \vocab{commutative ring} if $x \cdot  y= y \cdot  x$ for all $x,y \in R$. In this course, we will only consider commutative rings.
\end{definition*}
\begin{remarks}\leavevmode
       \begin{itemize}
             \item[(i)] As in the case of groups, we should check closure!
             \item[(ii)] For any $x \in R$, write $-x$ for the inverse of $x$ under $+$, and abbreviate $x+ (-y)$ as $x-y$.
             \item[(iii)] $0 \cdot x= (0+0)\cdot x=0=0 \cdot x+ 0 \cdot x $. Therefore $0 \cdot x=0$ for all $x \in R$.
             \item[(iv)] $0=0 \cdot  x=(1-1) \cdot  x= 1 \cdot x + (-1) \cdot x=x+ (-1)\cdot x$. So $(-1)\cdot  x=-x$ for all $x \in R$.
       \end{itemize}
\end{remarks}
\begin{definition*}[Subring]
       A subset $S \subset R$ is a subring (written $S \leq R$) if it is a ring under $+$ and $\cdot $, with the same identity elements 0 and 1.
\end{definition*}
\begin{example*}
       \leavevmode
       \begin{itemize}
             \item[(i)] $\mathbb{Z} \leq \mathbb{Q} \leq \mathbb{R} \leq \mathbb{C}$ are all rings.
             \item[(ii)] $\mathbb{Z}[i]=\left\{a+bi: \ a,b \in \mathbb{Z}\right\} \leq \mathbb{C}$ is a ring. This is called the ring of Gaussian integers.
             \item[(iii)] $\mathbb{Q}[\sqrt{2}]=\left\{a+b \sqrt{2}: \ a,b \in Q\right\}\leq \mathbb{R}$.
             \item[(iv)] $\mathbb{Z}/n\mathbb{Z}=$ (integers modulo $n$)
             \item[(v)] If $R,S$ are rings, we define the \vocab{product ring} to be the set $R \times S$ via \begin{align*}
                  (r_1 ,s_1 )+ (r_2 ,s_2 )=(r_1 +r_2 ,s_1 +s_2 )\\
                  (r_1 ,s_1 )\cdot  (r_2 ,s_2 )=(r_1 \cdot r_2 ,s_1 \cdot s_2 )\\
                  0_{R \times S}=(0_{R},0_{S}) \text{ and } 1_{R \times S}=(1_R,1_S).
             \end{align*}  
             \item[(vi)] For $R$ a ring, a \vocab{polynomial} $f$ over $R$ is an expression \[
             f=a_0 + a_1 X+a_2 X^2+ a_n X^{n}, \quad a_{i} \in R
             .\] "$X$" is just a formal symbol (i.e our definition of a polynomial just means some finite sequence in $R$). The degree of $f$ is the largest $n \in \mathbb{N}$ s.t $a_{n} \neq 0$. We write $R[X]$ for the set of all polynomials over $R$.

             If $g=b_0 +b_1 X + \ldots + b_{m}X^{m}$ is another polynomial, set 
             \begin{align*}
                   f+g= \sum_{i}^{}(a_{i}+b_{i})X^{i}\\
                   f \cdot g=\sum_{i}^{}(\sum_{j=0}^{i}a_{j}b_{i-j})X^{i}
             \end{align*}
             Then $R[X]$ is a ring with identities $0_{R}$ and $1_{R}$, which are constant polynomials. We identify $R$ with the subring of constant polynomials (i.e $a_{i}=0$ for all $i>0$)
       \end{itemize}
\end{example*}
\begin{definition*}[Unit]
       An element $r \in R$ is a \vocab{unit} if it has an inverse under multiplication, i.e $\exists s \in R: \ s \cdot r=1$.

       The units in $R$ form an abelian group $(R^{\times},\cdot )$ under multiplication. For example, $\mathbb{Z}^{\times}=\left\{\pm 1\right\}$ and $\mathbb{Q}^{\times}=\mathbb{Q}\backslash 0$. 
\end{definition*}
\begin{definition*}[Field]
       A \vocab{field} is a ring with $0 \neq 1$, such that every non-zero element is a unit. It's ``a ring where you can divide''. Examples of rings are $\mathbb{Q} $ and $\mathbb{Z}/p\mathbb{Z}$ for $p$ prime.
\end{definition*}
\begin{remark}
       If $R$ is a ring where $0=1$, then for all $x \in R$ we have \[
       x=1 \cdot x =0 \cdot x=0
       .\] So $R=\left\{0\right\}$ is the trivial ring. This is why we stipulate that $0 \neq 1$ for a ring to be a field.
\end{remark}
\begin{proposition}[Euclidean algorithm for rings]
       Let $f,g \in R[X]$. Suppose the leading coefficient of $g$ is a unit. Then there exist $q, r \in R[X]$ s.t. \[
       f (X)=q (X)g (X)+r (X) \quad \text{ where } \operatorname{deg}(r) < \operatorname{deg}(g)
       .\] 
\end{proposition}
\begin{proof}
       Induction on $n=\operatorname{deg}(f)$. Write \begin{align*}
             f (X)= a_{n}X^{n}+a_{n-1}X^{n-1}+ \ldots +a_0 , \quad a_{n}\neq 0\\
             g (X)= b_{n}X^{n}+b_{n-1}X^{n-1}+ \ldots +b_0 , \quad b_{n}\neq 0
       \end{align*}
       If $n<m$, then let $q=0$ and $r=f$. Done. Otherwise, we have $n \geq m$ and we set \[
       f_1 (X)= f (X)-a_{n} {b_{m}}^{-1} g (X) X ^{n-m}
       .\] The coefficient of $X^{n}$ is $a_{n}-a_{n}{b_{m}}^{-1}b_{m}=0$. Thus  $\operatorname{deg}(f_{1})<n$. 
       
       By the inductive hypothesis, $\exists q_1 ,r \in R[X]$ such that $f_1 (X)=q_1 (X)g (X)+r (X)$ where $\operatorname{deg}(r) < \operatorname{deg}(g)$. Therefore \[
       f (X)=\underbrace{q_1 (X)+a_n {b_{m}}^{-1} X ^{n-m}}_{q (X)}+r (X) 
       .\] So we are done. 
\end{proof}
\begin{example*}
       Let's look at further examples of rings.
       \begin{itemize}
             \item[(i)] If $R$ is a ring and $X$ is a set, then the set of all functions $X \rightarrow R$ is a ring under pointwise operations: \begin{align*}
                   (f+g)(x)= f (x)+g (x)\\
                   (f \cdot  g)(x)=f (x) \cdot  g (x)
             \end{align*} Further interesting examples appear as subrings, for example the ring of all $C^{1}$ functions from $\mathbb{R} \rightarrow \mathbb{R}$.
             \item[(ii)] 
       \end{itemize}
\end{example*}
\end{document}