\documentclass[a4paper]{scrartcl}

\usepackage[
    fancytheorems, 
    fancyproofs, 
    noindent, 
]{adam}
\usepackage{floatrow}
\usepackage{import}
\usepackage{xifthen}
\usepackage{pdfpages}
\usepackage{transparent}
\usepackage{bm}
\setcounter{section}{-1}
\newcommand{\incfig}[2]{%
    \def\svgwidth{#1mm}
    \import{./figures/}{#2.pdf_tex}
}

\title{IB Complex Methods}
\author{Martin von Hodenberg (\texttt{mjv43@cam.ac.uk})}
\date{\today}


\allowdisplaybreaks
\begin{document}

\maketitle

These are my notes for the IB course Complex Methods, which was lectured in Lent 2022 at Cambridge by Dr U. Sperhake. These notes are written in \LaTeX  \ for my own revision purposes. Any suggestions or feedback is welcome.



\tableofcontents
\newpage

\section{Background material}
\subsection{Complex numbers}
Recall the definition of a complex number, its real and imaginary parts, complex conjugate, modulus, and argument. Note that $\operatorname{arg} z$ is only defined up to adding $2n \pi$, for $n \in \mathbb{Z}$. Recall also the definition of the principal argument ($\operatorname{arg}z \in [- \pi, \pi]$). 

Recall the triangle inequality: \[
|z_1 |+|z_2 | \leq |z_1 |+ |z_2 | \quad \forall z_1 , z_2 \in \mathbb{C}
.\] 
By setting $z_1 = \zeta_1 +\zeta_2 $ and $z_2 =-\zeta_2 $ we get the reverse triangle inequality \[
||\zeta_1 |- |\zeta_2 ||\leq |\zeta_1 +\zeta_2 | \quad \forall \zeta_1 , \zeta_2 \in \mathbb{C}
.\] 

Recall the geometric series: for $z \in \mathbb{C}$, $z \neq 1$ and $n \in \mathbb{N}_0$: $\sum_{k=0}^{n}z^{k}= \frac{1-z^{n+1}}{1-z}$. 

For $|z|<1$, this converges for $n \rightarrow \infty$: $\sum_{k=0}^{ \infty}z^{k}= \frac{1}{1-z}$

\begin{definition*}[Open set]
     A set $D \subset \mathbb{C}$ is an "open set" if for all $z_0 \in D$, $\exists \varepsilon>0$ such that the $\varepsilon$-sphere $|z-z_0 |<\varepsilon$ lies in $D$. A neighbourhood of $z \in \mathbb{C}$ is an open set $D$ that contains $z$. 
\end{definition*}

\subsection{Trigonometric and hyperbolic functions}
Recall Euler's identity, and the complex definitions of cos, sin, and their hyperbolic counterparts. Recall that $\cos (ix)=\cosh (x)$  and $\sin (ix)= i \sinh (x)$ from the definitions.

\subsection{Calculus of real functions in $\geq 1$ variables}
Sometimes, we regard a complex function as 2 real functions on $\mathbb{R}^{2} : \ f (z)=u (x,y)+iv (x,y)$. (See IB Complex Analysis notes for more on this.)

\begin{definition*}
     We define $C^{m}(\Omega)$ as the set of functions $f: \Omega \subset \mathbb{R}^{n} \rightarrow \mathbb{R}$ whose partial derivatives up to order $m$ exist and are continuous. 
\end{definition*}
\begin{remark}
     We need the continuity condition: consider $f: \mathbb{R}^{2} \rightarrow \mathbb{R}$ defined by
     \begin{equation*}
        f (x,y)=
          \begin{cases}
              x & y=0\\
              y & x=0 \\
              1 & \text{elsewhere}
          \end{cases}      
     \end{equation*}
    Then $\frac{\partial f}{\partial x}(0,0)=1=\frac{\partial f}{\partial y}(0,0)$, but $f$ is not even continuous at $(0,0)$. 
\end{remark}

\begin{definition*}[Differentiable function]
     $f: \Omega \subset \mathbb{R}^{n} \rightarrow \mathbb{R}$ is differentiable at a point $x \in \Omega$ if there exists a linear function $A: \mathbb{R}^{n} \rightarrow R$ with \[
     f (x+h)-f (x)=A (x)(h) +o (||h||)
     .\] (See IB Analysis and Topology.) We define $f$ to be continuously differentiable if its partial derivatives are also continuous. This generalises to vector-valued functions $f: \Omega \rightarrow \mathbb{R}^{m}$ by considering each component $f_{i}$ separately.  
\end{definition*}

\begin{definition*}[Uniform convergence]
     A sequence of functions $f_{k}: \Omega \subset \mathbb{R}^{n} \rightarrow \mathbb{R}$ is uniformly convergent with limit $f$ iff \[
     \forall \varepsilon>0, \exists n \in \mathbb{N}: \quad  \forall k \geq n, x \in \Omega: \quad |f_{k} (x)- f (x)|< \varepsilon
     .\]
     See IB Analysis and Topology for more. In this course, we will use this to justify swapping limits with integrals and sums. 
\end{definition*}

\section{Analytic functions}
\subsection{The extended complex plane of the Riemann sphere}
We can identify $\mathbb{C}$ with $\mathbb{R}^2$ since $z \leftrightarrow (x,y)$ is bijective with $z=x+iy$.
\begin{definition*}[Extended complex domain]
     We define the extended complex domain $\mathbb{C}^*=\mathbb{C}\cup \infty$. We have seen this before in IA Groups. Some things to remember: 
     \begin{itemize}
          \item $z= \infty$ is a single point
          \item $z=- \infty$ means we approach this point along the negative real axis. 
          \item We can visualise this as the Riemann sphere: [image] We have the projection mapping $P \leftrightarrow z$ via the line $\vec{NP}$. The south pole $P$ is mapped to $z=0$ and the north pole $N$ is mapped to $z= \infty$. In practice, $f$ has a property at $z= \infty$ if $f \left(\frac{1}{\zeta}\right)$ has this property at $\zeta=0$.
     \end{itemize}
\end{definition*}
\subsection{Complex differentiation and analytic functions}
\begin{definition*}[Differentiability in $\mathbb{C}$]
      $f: \mathbb{C} \rightarrow \mathbb{C}$ is "differentiable" at $z \in \mathbb{C}$ if \[
      f' (z) \equiv \lim_{\delta z \rightarrow 0} \frac{f (z+ \delta z)-f (z)}{\delta z}
      \] exists and is independent of the direction of approach.
\end{definition*}
\begin{remark}
      Direction independence is a strong requirement! (See IB Complex Analysis for more). In $\mathbb{R}$, we have only 2 directions: for example, $f (x)=|x|$ is not differentiable at $x=0$. Conversely, in $\mathbb{C}$, we have infinitely many directions. 
\end{remark}
\begin{definition*}[Analyticity]
      A complex function $f$ is analytic at $z \in \mathbb{C}$ iff there exists a neighbourhood $D$ of $z$ where $f$ is differentiable.
\end{definition*}
\begin{remark}
      \begin{itemize}
           \item In this course, we will almost exclusively consider analytic functions.
           \item Analyticity implies many things: an analytic function can be differentiated infinitely many times, as opposed to in $\mathbb{R}$.
           \item Many rules of differentiation of real functions hold for complex ones, too. 
      \end{itemize}
\end{remark}
Let us consider 2 directions for $f (z)=u (x,y)+iv (x,y)$. 
\begin{enumerate}
     \item Real axis: $\delta z= \delta x$. So we have 
     \begin{align*}
          f' (z)&= \lim_{\delta x \rightarrow 0} \frac{f (z+\delta x)-f (z)}{\delta x}\\
          &=\lim_{\delta x \rightarrow 0} \frac{u (x+\delta x,y)+iv (x+\delta x,y)-u (x,y)-iv (x,y)}{\delta x}\\
          &=\frac{\partial u}{\partial x}+i \frac{\partial v}{\partial x}
     \end{align*}
     \item Imaginary axis: $\delta z= i \delta y$. So we have 
     \begin{align*}
          f' (z)&= \lim_{\delta y \rightarrow 0} \frac{f (z+i\delta y)-f (z)}{i \delta y}\\
          &=\lim_{\delta y \rightarrow 0} \frac{u (x,y+\delta y)+iv (x,y+ \delta y)-u (x,y)-iv (x,y)}{i \delta y}\\
          &=-i\frac{\partial u}{\partial y}+\frac{\partial v}{\partial y}
     \end{align*}
\end{enumerate}
But these two expressions must agree in their real and imaginary parts.
\begin{proposition}
      Any differentiable function $f (z)=u (x,y)+iv (x,y)$ must satisfy the Cauchy-Riemann equations \[
      \frac{\partial u}{\partial x}=\frac{\partial v}{\partial y}, \quad \frac{\partial u}{\partial y}= - \frac{\partial v}{\partial x}
      .\] The converse does not hold; we also need that $u,v$ are differentiable (see IB Complex Analysis)
\end{proposition}
In practice we use 
\begin{proposition}
      If $f (z)=u (x,y)+iv (x,y)$ saitsfies the Cauchy-Riemann equations at $z=z_0 $ and the partial derivatives are continuous in a neighbourhood of $z_0 $, then $f (z)$ is differentiable at $z_0 $. See IB Complex Analysis for the proof. 
\end{proposition}
\begin{proposition}[Product rule]
      The product of two analytic functions $f,g$ is analytic with \[
      (fg)' (z)=f' (z)g (z)+ f (z)g' (z)
      .\]
\end{proposition}
\begin{proof}
      Let us define 
      \begin{align*}
           &\overline{\nu} = \frac{f (z+h)-f (z)}{h}-f' (z)\\
           &\nu= \frac{g (z+h)-g (z)}{h}-g' (z)
      \end{align*}
      so both $\overline{\nu} ,\nu \rightarrow 0$ as $h \rightarrow 0$, and 
      \begin{align*}
           (gf)'&= \lim_{h \rightarrow 0} \frac{g (z+h)f (z+h)-g (z)f (z)}{h}\\
           &= \lim_{h \rightarrow 0} \frac{[g (z)+ (g' (z)+\nu)h][f (z)+(f' (z)+\overline{\nu} )h]-g (z)f (z)}{h}\\
           &= \lim_{h \rightarrow 0} \frac{(g' (z)+\nu)h f (z)+((f' (z)+\overline{\nu} )h g (z))+(g'(z)+\nu)h (f' (z)+\overline{\nu} )h     }{h}\\
           &=g' (z)f (z)+ g (z)f' (z).
      \end{align*}
\end{proof}
\begin{proposition}[Chain rule]
      The composition of two analytic functions $f,g$ is analytic with \[
      (f \circ g)' (z)= f' (g (z)) g' (z)
      .\] 
\end{proposition}
\begin{proof}
      Omitted.
\end{proof}
\begin{example*}
     Now let's look at some examples. 
      \begin{enumerate}
           \item $f (z)=z=x+iy$ is \vocab{entire}, i.e analytic in all of $\mathbb{C}$. This is true since $\frac{\partial u}{\partial x}=1= \frac{\partial v}{\partial y}, \quad \frac{\partial u}{\partial y}=0 =- \frac{\partial v}{\partial y}$ and they are continuous. The definition of $f' (z)$ gives us $f' (z)=1$.
           \item $e^{z}=e^{x+iy}$ 
      \end{enumerate}
\end{example*}
\end{document}