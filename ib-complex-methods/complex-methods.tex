\documentclass[a4paper]{scrartcl}

\usepackage[
    fancytheorems, 
    fancyproofs, 
    noindent, 
]{adam}
\usepackage{floatrow}
\usepackage{import}
\usepackage{xifthen}
\usepackage{pdfpages}
\usepackage{transparent}
\usepackage{bm}
\setcounter{section}{-1}
\newcommand{\incfig}[2]{%
    \def\svgwidth{#1mm}
    \import{./figures/}{#2.pdf_tex}
}

\title{IB Complex Methods}
\author{Martin von Hodenberg (\texttt{mjv43@cam.ac.uk})}
\date{\today}


\allowdisplaybreaks
\begin{document}

\maketitle

These are my notes for the IB course Complex Methods, which was lectured in Lent 2022 at Cambridge by Dr U. Sperhake. These notes are written in \LaTeX  \ for my own revision purposes. Any suggestions or feedback is welcome.



\tableofcontents
\newpage

\section{Background material}
\subsection{Complex numbers}
Recall the definition of a complex number, its real and imaginary parts, complex conjugate, modulus, and argument. Note that $\operatorname{arg} z$ is only defined up to adding $2n \pi$, for $n \in \mathbb{Z}$. Recall also the definition of the principal argument ($\operatorname{arg}z \in [- \pi, \pi]$). 

Recall the triangle inequality: \[
|z_1 |+|z_2 | \leq |z_1 |+ |z_2 | \quad \forall z_1 , z_2 \in \mathbb{C}
.\] 
By setting $z_1 = \zeta_1 +\zeta_2 $ and $z_2 =-\zeta_2 $ we get the reverse triangle inequality \[
||\zeta_1 |- |\zeta_2 ||\leq |\zeta_1 +\zeta_2 | \quad \forall \zeta_1 , \zeta_2 \in \mathbb{C}
.\] 

Recall the geometric series: for $z \in \mathbb{C}$, $z \neq 1$ and $n \in \mathbb{N}_0$: $\sum_{k=0}^{n}z^{k}= \frac{1-z^{n+1}}{1-z}$. 

For $|z|<1$, this converges for $n \rightarrow \infty$: $\sum_{k=0}^{ \infty}z^{k}= \frac{1}{1-z}$

\begin{definition}[Open set]
     A set $D \subset \mathbb{C}$ is an "open set" if for all $z_0 \in D$, $\exists \varepsilon>0$ such that the $\varepsilon$-sphere $|z-z_0 |<\varepsilon$ lies in $D$. A neighbourhood of $z \in \mathbb{C}$ is an open set $D$ that contains $z$. 
\end{definition}

\subsection{Trigonometric and hyperbolic functions}
Recall Euler's identity, and the complex definitions of cos, sin, and their hyperbolic counterparts. Recall that $\cos (ix)=\cosh (x)$  and $\sin (ix)= i \sinh (x)$ from the definitions.

\subsection{Calculus of real functions in $\geq 1$ variables}
Sometimes, we regard a complex function as 2 real functions on $\mathbb{R}^{2} : \ f (z)=u (x,y)+iv (x,y)$. (See IB Complex Analysis notes for more on this.)

\begin{definition}
     We define $C^{m}(\Omega)$ as the set of functions $f: \Omega \subset \mathbb{R}^{n} \rightarrow \mathbb{R}$ whose partial derivatives up to order $m$ exist and are continuous. 
\end{definition}
\begin{remark}
     We need the continuity condition: consider $f: \mathbb{R}^{2} \rightarrow \mathbb{R}$ defined by
     \begin{equation*}
        f (x,y)=
          \begin{cases}
              x & y=0\\
              y & x=0 \\
              1 & \text{elsewhere}
          \end{cases}      
     \end{equation*}
    Then $\frac{\partial f}{\partial x}(0,0)=1=\frac{\partial f}{\partial y}(0,0)$, but $f$ is not even continuous at $(0,0)$. 
\end{remark}

\begin{definition}[Differentiable function]
     $f: \Omega \subset \mathbb{R}^{n} \rightarrow \mathbb{R}$ is differentiable at a point $x \in \Omega$ if there exists a linear function $A: \mathbb{R}^{n} \rightarrow R$ with \[
     f (x+h)-f (x)=A (x)(h) +o (||h||)
     .\] (See IB Analysis and Topology.) We define $f$ to be continuously differentiable if its partial derivatives are also continuous. This generalises to vector-valued functions $f: \Omega \rightarrow \mathbb{R}^{m}$ by considering each component $f_{i}$ separately.  
\end{definition}

\begin{definition}[Uniform convergence]
     A sequence of functions $f_{k}: \Omega \subset \mathbb{R}^{n} \rightarrow \mathbb{R}$ is uniformly convergent with limit $f$ iff \[
     \forall \varepsilon>0, \exists n \in \mathbb{N}: \quad  \forall k \geq n, x \in \Omega: \quad |f_{k} (x)- f (x)|< \varepsilon
     .\]
     See IB Analysis and Topology for more. In this course, we will use this to justify swapping limits with integrals and sums. 
\end{definition}

\section{Analytic functions}
\subsection{The extended complex plane of the Riemann sphere}
We can identify $\mathbb{C}$ with $\mathbb{R}^2$ since $z \leftrightarrow (x,y)$ is bijective with $z=x+iy$.
\end{document}