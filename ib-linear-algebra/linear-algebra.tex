\documentclass[a4paper]{scrartcl}

\usepackage[
    fancytheorems, 
    fancyproofs, 
    noindent, 
]{adam}
\usepackage{floatrow}
\usepackage{import}
\usepackage{xifthen}
\usepackage{pdfpages}
\usepackage{transparent}

\newcommand{\incfig}[2]{%
    \def\svgwidth{#1mm}
    \import{./figures/}{#2.pdf_tex}
}

\title{IB Linear Algebra (from lecture 18)}
\author{Martin von Hodenberg (\texttt{mjv43@cam.ac.uk})}
\date{\today}


\allowdisplaybreaks

\begin{document}

\maketitle

Linear algebra description etc

This article constitutes my notes for the `IB Linear Algebra' course, held in Michaelmas 2021 at Cambridge. The course was lectured by Prof. Pierre Raphael.


\tableofcontents

\section{Bilinear forms}
\begin{lemma}
    We have a bilinear form $\phi: V \times V \rightarrow \mathbb{F}$, where $V$ is a finite vector space and $B, B'$ are bases of $V$. Let 
    \[\phi=[Id]_{B,B'}.\]
     Then 
    \[[\phi]_{B'}=P^T [\phi]_B P.\]
\end{lemma}
\begin{proof}
     This is just a special case of the general change of basis formula; see that proof.
\end{proof}

\begin{definition}[Congruent matrices]
     Two square matrices $A,B$ are said to be \vocab{congruent} if there exists an invertible square matrix $P$ such that 
     \[A=P^T B P.\]
\end{definition}
\begin{remark}
     This defines an equivalence relation.
\end{remark}
\begin{definition}[Symmetric bilinear form]
     A bilinear form on $V$ is said to be \vocab{symmetric} if 
     \[\phi(u,v)=\phi(v,u) \ \forall u,v \in V.\]
\end{definition}
\begin{remark}
    \begin{enumerate}
        \item If $A$ is a square matrix, we say that $A$ is symmetric if $A^T=A$. Equivalently, $A_{ij}=A_{ji}$.
        \item $\phi$ is symmetric iff $[\phi]_B$ is symmetric in \emph{any} basis $B$.
        \item To be able to represent $\phi$ by a diagonal matrix in some basis $B$, it is necesssary that $\phi$ is symmetric: 
        \[P^T AP=D=D^T=P A^T P^T \implies A=A^T \implies \phi \text{is symmetric} .\]
        
    \end{enumerate}
     
\end{remark}
\begin{definition}[Quadratic form]
     A map $Q: V \rightarrow F$ is said to be a \vocab{quadratic form} if there exists a bilinear form $\phi: V \times V \rightarrow F$ such that 
     \[\forall u \in V, \ Q(u)\phi (u,u).\]
\end{definition}
\begin{remark}[Computation in a basis]
     Let $B=(e_i)_{1 \leq i \leq n}$ be a basis of $V$, and let $A=[\phi]_B$. Let $u=\sum_{i=1}^{n}u_i e_i$, then 
     \[Q (u)-\phi (u,u)=\phi (\sum_{i=1}^{n}u_i e_i, \sum_{j=1}^{n})u_j e_j=\sum_{i,j=1}^{n}u_i u_j e_i e_j=\sum_{i,j=1}^{a_ij u_i u_j}.\]
     (by bilinearity of $\phi$)
     Therefore we essentially have 
     \[Q (u)=U^T A U \text{, where } U=\begin{pmatrix}
     u_1\\\vdots \\ u_n
     \end{pmatrix}
     .\]
\end{remark}
\begin{remark}
     We can note that 
     \[Q (u)=U^T A U =\sum_{i,j=1}^{n}a_{ij}u_i u_j=\sum_{i,j=1}^{n}(\frac{a_{ij}+a_{ji}}{2})u_i u_j=U^T (\frac{A+A^T}{2}) U.\]
     So the representation of $A$ is not necessarily unique.
\end{remark}

\begin{proposition}
     If $Q: V \rightarrow F$ is a quadratic form, then there exists a unique symmetric bilinear form $\phi: V \times V \rightarrow F $ such that 
     \[Q(u)=\phi (u,u) \forall u \in V.\]
\end{proposition}
\begin{proof}[Polarisation identity]
     Let $\psi$ be a bilinear form on $V$ such that 
     \[\forall u \in V, Q(u)=\psi (u,u).\]
     Let $\phi (u,v)=\frac{1}{2} (\psi (u,v)+ \psi (v,u))$. Thus we have that:
     \begin{itemize}
         \item $\phi$ is a bilinear form
         \item $\phi$ is symmetric
         \item $\phi (u,u)=\psi (u,u)=Q (u)$.
     \end{itemize}
     This concludes the proof of \emph{existence}.\newline 
     \textbf{Proof of uniqueness}\newline 
     Let $\phi$ be a symmetric bilinear form such that 
     \[\forall u \in V, \phi (u,u)=Q(u).\]
     Then 
     \begin{equation*}
          \begin{split}
            Q (u+v)&=\phi (u+v,u+v)\\
            &=\phi (u,u)+ \phi (u,v)+\phi (v,u)+ \phi (v,v) \text{ by bilinearity} \\
            &=Q(u)+ 2\phi (u,v)+Q(v) \text{ by symmetry}\\
          \end{split}
     \end{equation*}
     From this we get that 
     \[\phi(u,v)=\frac{1}{2} (Q (u+v)-Q (u)- Q (v)).\]
\end{proof}

\begin{theorem}[Diagonalisation of symmetric bilinear forms]
     Let $\phi: V \times V \rightarrow F$ be a symmetric bilinear form. (dim $V$=n). Then there exists a basis $B$ of $V$ such that $[\phi]_B$ is diagonal.
\end{theorem}
\begin{proof}
     We proceed by induction on the dimension of $V$. For $n=1$ it is trivially true. Suppose the theorem holds for all dimensions $<n$: then
     \begin{itemize}
         \item If $\phi (u,u)=0 \forall u \in V$, then $\phi=0$ by the polarisation identity ($\phi $ is symmetric).
         \item If $\phi \neq 0$, then there exists a $u \in V \backslash \{0\}$ such that $\phi (u,u) \neq 0$. Let us call $u=e_1$.
         \item Let $U$ be the 'orthogonal' of $e_1$: 
         \[U=(<e_1>)=\{v \in V: \ \phi (e_1,v)=0\}=\operatorname{ker} \theta: v \rightarrow \phi (e_1,v).\]
         Since it is a kernel of a linear map $V \rightarrow F$, therefore $U$ is a vector subspace of $V$. By the Rank-Nullity theorem, we have 
         \[\operatorname{dim} V=n= R(\theta)+ \operatorname{null}  \theta=\operatorname{dim}U+1.\]
         We now claim that $U+<e_1>=U \oplus <e_1>$. Indeed, 
         \[v= <e_1> \cap U \implies v= \lambda e_1, \phi(e_1,v)=0 .\]
         \[\implies 0=\phi (e_1,v)=\phi (e_1, \lambda e_1)=\lambda \phi (e_1,e_1) \implies \lambda=0 \implies v=0.\]
         
         \[\implies U+<e_1>=U \oplus <e_1>.\]
         Therefore $V=U \oplus <e_1>$, and pick a basis $B'=(e_2, \ldots ,e_n)$ such that $(e_1, e_2, \ldots , e_n)$ is a basis of $V$ (since the sum is direct). So 
         \[[\phi]_B=(\phi(e_i,e_j))_{1 \leq i,j \leq n}=\begin{pmatrix}
         \phi (e_1,e_1) &0\\0&A'
         \end{pmatrix}
         .\]
         Therefore $(A')^T=A'$, and 
         $A'=[\phi|_U]_{B'}$ where $\phi|_U$ is the restriction of $\phi$ onto $U$. Now we apply the induction hypothesis to find a basis $(e_1', \ldots ,e_n;)$ of $V$ such that $[\phi|U]_{B}$ is diagonal. So 
         \[\hat{B}=(e_1,e_2',\ldots ,e_n')\] is a basis of $V$, and finally we have that $[\phi]_{\hat{B}}$ is diagonal.      
     \end{itemize}
\end{proof}

\begin{example}
     Let $V=\mathbb{R}^{3}$, and 
     \[Q(x_1,x_2,x_3)=x_1^2+x_2^2+2x_3^2+2x_1x_2+2x_1x_3+2x_2x_3.\]
     
\end{example}
\end{document}