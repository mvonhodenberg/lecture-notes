\documentclass[a4paper]{scrartcl}

\usepackage[
    fancytheorems, 
    fancyproofs, 
    noindent, 
]{adam}
\usepackage{floatrow}
\usepackage{import}
\usepackage{xifthen}
\usepackage{pdfpages}
\usepackage{transparent}
\usepackage{bm}

\newcommand{\incfig}[2]{%
    \def\svgwidth{#1mm}
    \import{./figures/}{#2.pdf_tex}
}

\title{IB Fluid Dynamics}
\author{Martin von Hodenberg (\texttt{mjv43@cam.ac.uk})}
\date{\today}
\setcounter{section}{-1}

\allowdisplaybreaks

\begin{document}

\maketitle


These are my notes for the IB course Fluid Mechanics, which was lectured in Lent 2022 at Cambridge by Prof. J.R. Lister. These notes are written in \LaTeX  \ for my own revision purposes. Any suggestions or feedback is welcome.



\tableofcontents
\newpage
\section{Introduction}
Fluid mechanics is a subset of a field called continuum mechanics. Flows of liquids and gases are both fluids, and continuum mechanics also includes solid mechanics such as elasticity and deformation. 

What we are interested in in this course is the overall behaviour of a fluid at a macroscopic level. We aim to take averages over irrelevant molecular details to get a continuous description in terms of fields; e.g density $\rho (x,t)$ or velocity $u (x,t)$.

We talk about \vocab{steady flows } $u=u (x)$, and \vocab{unsteady flows} $u=u (x,t)$. We combine simple physics (mass, momentum, Newton's laws) from IA Dynamics and Relativity with knowledge from IA Vector Calculus and IB Methods. Our goal is to find the flow $u (x,t)$ and the accompanying forces. 

We will look at \vocab{inviscid flow} (very good for water and air, except on very small scales) and \vocab{simple viscous flow} - more on this in Part II Fluid Dynamics and Part II Waves.

Fluids are everywhere, and as such this course has a wide range of applications (environmental science, biology, \ldots)

\section{Kinematics}
There are two natural perspectives on flow:
\begin{enumerate}
    \item A stationary observer watching the flow go past (\emph{Eulerian} viewpoint)
    \item A moving observer travelling along with the flow (\emph{Lagrangian} viewpoint)
\end{enumerate}
\subsection{Streamlines}
\begin{definition}[Streamline]
     A \vocab{streamline} is a curve that is everywhere parallel to the flow at a given instant. The streamline through $\bm{x_0 } $ at time $t_0 $ can be found parametrically in the form \[
     \bm{x} =\bm{x} (s; \bm{x_0 } , \bm{t_0 } )
     .\] from solving $\frac{\mathrm{d}x}{\mathrm{d}s}=\bm{u} (\bm{x};t_0 )$ with $\bm{x} =\bm{x_0 } $ at $s=0$. The set of streamlines at a given instant shows the direction of the flow.
\end{definition}
For example, for a flow $u= (1,t)$ we would solve \[
\int x = x_0 +s, y=y_0 +ts \implies y=y_0 +t (x-x_0 )
.\] 
\subsection{Pathlines / Particle paths}
\begin{definition}[Pathline/Particle path]
     A \vocab{pathline} is the trajectory of a fluid "particle" (i.e a very small blob of fluid). The pathline $x=x (t;x_0 )$ of the fluid particle at $x=x_0 $ when $t=0$ is given by $\frac{\mathrm{d}x}{\mathrm{d}t}=u (x,t)$ with $x=x_0 $ at $t=0$.
\end{definition}
In our previous example $u= (1,t)$ this would give 
\begin{align*}
    x=x_0 +t, \quad y=y_0 +t^2 \\
    \implies y=y_0 + \frac{1}{2} (x-x_0)^2.
\end{align*}
This Lagrangian viewpoint is often more complicated than the Eulerian one, but we can for example consider all $x_0 $ in some region $\mathcal{D}$ to describe how the shape and position of a dyed patch of fluid evolves. Useful for thinking about transport and mixing.
\begin{remark}
     For steady flow only, pathlines and streamlines are the same. 
\end{remark}
\begin{definition}[Material derivative]
     The \vocab{material derivative} is the rate of change moving with the fluid. [picture] For any quantitity $F (x,t)$, the rate of change seen by an observer with the flow is found from \[
     \delta F =F (x+\delta x, t+\delta t)-F (x,t)=\delta x \cdot \nabla F +\delta t \frac{\partial F}{\partial t}+o (t)
     .\] 
     The displacement of fluid moving with the flow is given by \[
     \delta x= u (x,t) \delta t + o (t)
     .\] Therefore dividing by $\delta t$ and taking the limit as $\delta \rightarrow 0$, we get the material derivative \[
     \frac{\mathcal{D}F}{\mathcal{D}t}=\underbrace{\frac{\partial F}{\partial t}}_{\text{Eulerian deriv.} } + \underbrace{u \cdot \nabla F}_{\text{convected derivative} } 
     .\] 
\end{definition}
\end{document}