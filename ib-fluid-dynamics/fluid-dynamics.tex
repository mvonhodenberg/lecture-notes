\documentclass[a4paper]{scrartcl}

\usepackage[
    fancytheorems, 
    fancyproofs, 
    noindent, 
]{adam}
\usepackage{floatrow}
\usepackage{import}
\usepackage{xifthen}
\usepackage{pdfpages}
\usepackage{transparent}
\usepackage{bm}

\newcommand{\incfig}[2]{%
    \def\svgwidth{#1mm}
    \import{./figures/}{#2.pdf_tex}
}
\renewcommand{\vec}[1]{\underline{#1}}

\title{IB Fluid Dynamics (up to lecture 2)}
\author{Martin von Hodenberg (\texttt{mjv43@cam.ac.uk})}
\date{\today}
\setcounter{section}{-1}

\allowdisplaybreaks

\begin{document}

\maketitle


These are my notes for the IB course Fluid Mechanics, which was lectured in Lent 2022 at Cambridge by Prof. J.R. Lister. These notes are written in \LaTeX  \ for my own revision purposes. Any suggestions or feedback is welcome.



\tableofcontents
\newpage
\section{Introduction}
Fluid mechanics is a subset of a field called continuum mechanics. Flows of liquids and gases are both fluids, and continuum mechanics also includes solid mechanics such as elasticity and deformation. 

What we are interested in in this course is the overall behaviour of a fluid at a macroscopic level. We aim to take averages over irrelevant molecular details to get a continuous description in terms of fields; e.g density $\rho (x,t)$ or velocity $u (x,t)$.

We talk about \vocab{steady flows } $u=u (x)$, and \vocab{unsteady flows} $u=u (x,t)$. We combine simple physics (mass, momentum, Newton's laws) from IA Dynamics and Relativity with knowledge from IA Vector Calculus and IB Methods. Our goal is to find the flow $u (x,t)$ and the accompanying forces. 

We will look at \vocab{inviscid flow} (very good for water and air, except on very small scales) and \vocab{simple viscous flow} - more on this in Part II Fluid Dynamics and Part II Waves.

Fluids are everywhere, and as such this course has a wide range of applications (environmental science, biology, \ldots)

\section{Kinematics}
There are two natural perspectives on flow:
\begin{enumerate}
    \item A stationary observer watching the flow go past (\emph{Eulerian} viewpoint)
    \item A moving observer travelling along with the flow (\emph{Lagrangian} viewpoint)
\end{enumerate}
\subsection{Particle paths and streamlines}
\begin{definition*}[Streamline]
     A \vocab{streamline} is a curve that is everywhere parallel to the flow at a given instant. The streamline through $\bm{x_0 } $ at time $t_0 $ can be found parametrically in the form \[
     \bm{x} =\bm{x} (s; \bm{x_0 } , \bm{t_0 } )
     .\] from solving $\frac{\mathrm{d}x}{\mathrm{d}s}=\bm{u} (\bm{x};t_0 )$ with $\bm{x} =\bm{x_0 } $ at $s=0$. The set of streamlines at a given instant shows the direction of the flow.
\end{definition*}
For example, for a flow $u= (1,t)$ we would solve \[
\int x = x_0 +s, y=y_0 +ts \implies y=y_0 +t (x-x_0 )
.\] 
\begin{definition*}[Pathline/Particle path]
     A \vocab{pathline} is the trajectory of a fluid "particle" (i.e a very small blob of fluid). The pathline $x=x (t;x_0 )$ of the fluid particle at $x=x_0 $ when $t=0$ is given by $\frac{\mathrm{d}x}{\mathrm{d}t}=u (x,t)$ with $x=x_0 $ at $t=0$.
\end{definition*}
In our previous example $u= (1,t)$ this would give 
\begin{align*}
    x=x_0 +t, \quad y=y_0 +t^2 \\
    \implies y=y_0 + \frac{1}{2} (x-x_0)^2.
\end{align*}
This Lagrangian viewpoint is often more complicated than the Eulerian one, but we can for example consider all $x_0 $ in some region $\mathcal{D}$ to describe how the shape and position of a dyed patch of fluid evolves. Useful for thinking about transport and mixing.
\begin{remark}
     For steady flow only, pathlines and streamlines are the same. 
\end{remark}
\begin{definition*}[Material derivative]
     The \vocab{material derivative} is the rate of change moving with the fluid. [picture] For any quantitity $F (x,t)$, the rate of change seen by an observer with the flow is found from \[
     \delta F =F (x+\delta x, t+\delta t)-F (x,t)=\delta x \cdot \nabla F +\delta t \frac{\partial F}{\partial t}+o (t)
     .\] 
     The displacement of fluid moving with the flow is given by \[
     \delta x= u (x,t) \delta t + o (t)
     .\] Therefore dividing by $\delta t$ and taking the limit as $\delta \rightarrow 0$, we get the material derivative \[
     \frac{\mathcal{D}F}{\mathcal{D}t}=\underbrace{\frac{\partial F}{\partial t}}_{\text{Eulerian deriv.} } + \underbrace{u \cdot \nabla F}_{\text{convected derivative} } 
     .\] 
\end{definition*}
\subsection{Conservation of mass}
Consider an arbitrary volume $V$, fixed in space, bounded by a surface $\partial V$ with outward normal $\vec{n}$ with a flow of mass across it.

The mass $\int_{V}^{}\rho  \mathrm{d}V $ inside $V$ can only change due to the flow of mass across the boundary $\partial V$.
[picture]
Volume out over $\partial S$ in time $\delta t$ =$(\vec{u}\cdot \vec{n})\delta S \delta t$. Therefore the mass out is $\rho(\vec{u}\cdot \vec{n})\delta S \delta t$. Thus
\begin{align*}
    &\frac{\mathrm{d}}{\mathrm{d}t} \int_{V}^{}\rho \mathrm{d}V =- \int_{\partial V}^{}\rho (\vec{u}\cdot \vec{n}) \mathrm{d}S\\
    \text{V fixed} \implies & \int_{V}^{}\frac{\partial \rho}{\partial t} \mathrm{d} V=-\int_{V}^{}\nabla \cdot (\rho \vec{u}) \mathrm{d}V \text{ by divergence theorem}\\ 
    \text{true for any } V \implies & \frac{\partial \rho}{\partial t} +\nabla \cdot (\rho \vec{u})=0.
\end{align*}
We say that $\rho u$ is the \vocab{mass flux}. The density at a point changes due to the divergence/convergence of the mass flux. Using $\nabla \cdot (\rho \vec{u})= \vec{u}\cdot \nabla \rho + \rho \nabla \cdot \vec{u}$ we can rewrite this as \[
\frac{\mathcal{D}\rho}{\mathcal{D}t}=-\rho \nabla \cdot \vec{u}
.\] The density of a blob of fluid decreases if the velocity is divergent (spreading out). 

If a fluid is incompressible, i.e the density of a fluid particle cannot change, then $\frac{\mathcal{D}\rho}{\mathcal{D}t}= 0 \implies \nabla \cdot \vec{u} =0$. This course will consider fluids with a constant uniform density $\rho$. 
\subsection{Kinematic boundary conditions}
Suppose the material boundary of a body of fluid has velocity $\vec{U} (\vec{x},t)$. At a point $\vec{x}$ on the boundary, the fluid velocity relative to the moving boundary is $\vec{u} (\vec{x},t)-\vec{U} (\vec{x}, t)$. The condition that there is no mass flux across the boundary is \[
\rho (\vec{u}-\vec{U})\cdot \vec{n} \delta S \delta t=0 \quad \implies \vec{u}\cdot \vec{n}=\vec{U}\cdot \vec{n}
.\] We can use this in several examples: 
\begin{enumerate}
    \item For example at a fixed boundary, $\vec{U}=0$. This implies $\vec{u} \cdot \vec{n}$.
    \item Water waves (chapter 5) have water/air interface $z=\zeta (x,y,t)$. If the water surface has normal $\vec{n}$ and we think of it as a contour of $F (x,y,z,t)=z- \zeta (x,y,t)$, then $\vec{n}$ is parallel to \[
    \nabla F= (-\zeta_{x},-\zeta_{y}, 1 )
    .\] We can take $\vec{U}=(0,0,\zeta_{t})$ and $\vec{u}=(u,v,w)$; then \[
    -u \zeta_{x}-v \zeta_{y}+w= \zeta_{t}
    .\] This equation is equivalent to $ \frac{\mathcal{D}\zeta}{\mathcal{D}t}=w$, i.e fluid blobs on the interface change height at rate $w$. 
\end{enumerate}
\subsection{Streamfunction for 2D incompressible flow}
Recall incompressible $\implies \nabla \cdot \vec{u}=0 \iff \vec{u}=\nabla \times \vec{A}$ for some vector potential $\vec{A}$ (IA Vector Calculus). Now if we restrict ourselves to 2D flow, we can write $\vec{u}=(u (x,y), v (x,y),0)$ and take $\vec{A}=(0,0,\psi (x,y))$. By taking the curl of $A$ we get
\begin{align*}
    \implies & \left(\frac{\partial \psi}{\partial y}, -\frac{\partial \psi}{\partial x},0\right)\\
    \implies & \nabla \cdot \vec{u}=\frac{\partial u}{\partial x}+\frac{\partial v}{\partial y}=0 \text{ as expected } .
\end{align*}
The scalar $\psi (x,y)$ is the \vocab{streamfunction}. This has several properties, which we will prove on Example Sheet 1: 
\begin{itemize}
    \item[(i)] The streamlines are given by $\psi=$ const (parallel to the contours of $\psi$)
    \item[(ii)] $|\vec{u}|=|\nabla \psi|$ so the flow is faster when the streamlines are closer. 
    \item[(iii)] $\psi (\vec{x_0})-\psi (\vec{x_1})=\int_{\vec{x_0}}^{\vec{x_1}}\vec{u} \cdot \vec{n} \mathrm{d}l $ which is the flux crossing the line $\vec{x_0} \rightarrow \vec{x_1}$.
    \item[(iv)] $\psi$ is a constant or a stationary rigid boundary. (kinematic boundary condition and (iii)).  
\end{itemize}
\begin{example*}
     Let $\vec{u}=(y,x)$. Then $\nabla \cdot \vec{u}=0$ so $\psi$exists. 
     \begin{align*}
        &\frac{\partial v}{\partial y}=y \implies \psi=\frac{1}{2} y^{2}+f (x)\\
        &-\frac{\partial v}{\partial x}=x  \implies f=-\frac{1}{2}x^2.
     \end{align*}
     The streamlines are $\frac{1}{2}( y^2-x^2)=$ const. 
\end{example*}
\subsubsection{Polar coordinates}
Let us express $\vec{u}$ and $\vec{A}$ in polar coordinates: \[
\vec{u}=(u_{r}(r,\theta), u_{v}(r, \theta),0) \quad \vec{A}=(0,0,\psi (r,\theta))
.\] 
Therefore 
\begin{align*}
    \implies \vec{u}=\nabla \times \vec{A}=(\frac{1}{r} \frac{\partial \psi}{\partial \theta}, -\frac{\partial \psi}{\partial r},0)\\
    \implies \nabla \cdot \vec{u}= \frac{1}{r} \frac{\partial}{\partial r}(ru_{r})+\frac{1}{r} \frac{\partial u_{\theta}}{\partial \theta}=0, \text{ as expected } 
\end{align*}
Axisymmetric flows in spherical polars have $u_{\phi}=0, \frac{\partial }{\partial \phi}=0$. If $\nabla \cdot \vec{u}=0$ and we take $\vec{A}=(0,0, \frac{\Psi (r, \theta)}{r \sin \theta})$, then \[
\vec{u}=\left(\frac{1}{r^2 \sin \theta} \frac{\partial \psi}{\partial \theta}, -\frac{1}{r \sin \theta}\frac{\partial \psi}{\partial r},0 \right)
.\] So $\vec{u}$ is parallel to the contours of the Stokes stream function $\Psi (r,\theta)$.
\section{Dynamics of inviscid flow}
In this chapter, we move onto dynamics, thinking about forces and accelerations. We will focus on inviscid fluids for this chapter.
\subsection{Surface and volume forces}
Two types of force act on a fluid:
\begin{itemize}
    \item[(i)] Those proportional to the volume, like gravity. We call these \vocab{volume forces}.
    \item[(ii)] Those proportional to the area, like pressure. We call these \vocab{surface forces}. 
\end{itemize}
\begin{definition*}[Volume force]
     We denote the force on a small volume element $\delta V$ by $\vec{F} (\vec{x},t)\delta V$. $\vec{F}$ is often \textbf{conservative} with potential energy/ unit volume $\chi$ and $F=-\nabla \chi$.

     The most common case for us is $\vec{F} \delta V$=$\rho \vec{g} \delta V$ with $\chi=pgz$ if $\vec{g}=(0,0-g)$. We write $p \vec{g}$ for the weight/unit volume.
\end{definition*}
\begin{definition*}[Surface force]
     Consider a small element of area $\vec{n} \delta A$. Here $\vec{n}$ is the direction of the force. Denote the surface force exerted by the + side on the - side by \[
     \vec{\tau} (\vec{x},t,\vec{n})\delta A
     .\] where $\vec{\tau}$ is called the \vocab{stress} acting on the area element. (A stress is a force per unit area.)
\end{definition*}
\begin{remark}
     The stress $\vec{\tau}$ depends on the orientation of the surface $\vec{n}$. For example, by Newton's third law, the force exerted by the - side on the + side is $-\vec{\tau}$.
\end{remark}
We defer any discussion of viscous stresses (e.g. friction) to Section 3. In many phenomena, the viscous stresses are negligible, and the fluid behaves as if it is \vocab{inviscid} (frictionless). In this model, the stress $\vec{t}$ acting on $\vec{n}\delta A$ has no tangential component, and it has a magnitude independent of orientation \[
\vec{t} (\vec{x}, t , \vec{n})=- p (\vec{x},t) \vec{n}
.\] where $p$ is the pressure.
\subsection{The Euler equation}
Consider an arbitrary volume $V$, fixed in space, bounded by a surface $\partial V$ with outward normal $\vec{n}$. The momentum inside $V$ is \[
\int_{V}^{}p \vec{u} \ \mathrm{d}dV 
.\] and can change due to 
\begin{itemize}
    \item[(i)] Flow of momentum across the bounding surface $\partial V$. 
    \item[(ii)] Volume/body forces.
    \item[(iii)] Surface forces.
\end{itemize}
\end{document}