\documentclass[a4paper]{scrartcl}

\usepackage[
    fancytheorems, 
    fancyproofs, 
    noindent, 
]{adam}
\usepackage{floatrow}
\usepackage{import}
\usepackage{xifthen}
\usepackage{pdfpages}
\usepackage{transparent}
\usepackage{bm}

\newcommand{\incfig}[2]{%
    \def\svgwidth{#1mm}
    \import{./figures/}{#2.pdf_tex}
}

\title{IB Complex Analysis}
\author{Martin von Hodenberg (\texttt{mjv43@cam.ac.uk})}
\date{\today}


\allowdisplaybreaks
\begin{document}

\maketitle
These are my notes for the IB course Complex Analysis, which was lectured in Lent 2022 at Cambridge by Prof. N.Wickramasekera. These notes are written in \LaTeX  \ for my own revision purposes. Any suggestions or feedback is welcome.

\tableofcontents
\newpage

\section{Basic notions}
Recall definitions in $\mathbb{C}$ from IA courses. Note that $d (z,w)=|z-w|$ defines a metric on $\mathbb{C}$ (the standard metric). For $a \in \mathbb{C}$ and $r >0$, we write $D (a,r)= \left\{z \in \mathbb{C}: \ |z-a|<r \right\}$ for the open ball with centre $a$ and radius $r$. 
\begin{definition*}[Open subset of $\mathbb{C}$]
     A subset $U \subset \mathbb{C}$ is open wrt. the standard metric if for all $a \in U$, there exists an $r$ such that $D (a,r) \in U$.
\end{definition*} 
\begin{remark}
     This is equivalent to being open wrt. the Euclidean metric on $\mathbb{R}^{2} $.
\end{remark}

This course is about complex valued functions of a single variable, i.e functions \[
f: A \rightarrow \mathbb{C}, \quad \text{where } A \subset \mathbb{C}
.\] 
Identifying $\mathbb{C}$ with $\mathbb{R}^2$ in the usual way, we can write $f=u+iv$ for real functions $u,v$ and thus define $u= \operatorname{Re} (f)$, $v=\operatorname{Im} (f)$. 
Almost exclusively we'll focus on differentiable functions $f$. But first let's recall continuity. 
\begin{definition*}[Continuous function on $\mathbb{C}$]
     The function $f$ (as above) is continuous at a point $w \in A$ if \[
     \forall \varepsilon >0, \ \exists \delta >0 \text{ such that } \forall z \in A,\quad  |z-w|<\delta \implies |f (z)- f (w)| <\varepsilon
     .\] 
\end{definition*}
\begin{remark}
     This is equivalent to saying that $\lim_{z \rightarrow w} f(z)=f (w) $. 
\end{remark}

\subsection{Complex differentiation}
Let $f: U \rightarrow \mathbb{C}$, where $U$ is open. 
\begin{definition*}[Differentiability]
     $f$ is differentiable at $w \in U$ if the limit \[
     f' (w)=\lim_{z \rightarrow w} \frac{f (z)-f (w)}{z-w}
     .\] exists at a complex number.
\end{definition*}
\begin{definition*}[Holomorphic function]
     $f$ is \vocab{holomorphic}\footnotemark at $w \in U$ if there is $\varepsilon >0$ such that $D (w, \varepsilon)\subset U$ and $f$ is differentiable at every point in $D (w, \varepsilon)$. 

     Equivalently, $f$ is holomorphic in $U$ if $f$ is holomorphic at every point in $U$, or equivalently, $f$ is differentiable at every point in $U$.
\end{definition*}\footnotetext{Sometimes we use "analytic" to mean holomorphic.}
Usual rules of differentiation of real functions of a real variable hold for complex functions. Derivatives of sums, products, quotients of functions are obtained in the same way (can easily be checked). 

\begin{proposition}
     The chain rule for composite functions also holds: if $f: U \rightarrow \mathbb{C}$, $g: V \rightarrow \mathbb{C}$ with $f (U) \subset V$, and $h= g \circ f : U \rightarrow \mathbb{C}$. If $f$ is differentiable at $w \in U$ and $g$ is differentiable at $f (w)$, then $h$ is differentiable at $w$ with \[
     h' (w)=(g \circ f)' (w)= f' (w) (g' \circ f)(w)
     .\] 
\end{proposition}
\begin{proof}
     Omitted; analagous to the proof for the real case.
\end{proof}
We might ask ourselves a question: 

Write $f (z)= u (x,y)+iv (x,y)$, $z=x+iy$. Is differentiability of $f$ at a point $w=c+id \in U$ is the same as differentiability of $u $ and $v$ at $(c,d)$? 

Recall from IB Analysis \& Topology that $u : U \rightarrow \mathbb{R}$ is differentiable at $(c,d) \in U$ if there is a "good linear approximation of $u$ at $(c,d)$". We can show that if  $u$ is differentiable at $c,d$ then $L$ (the derivative of $u$ at $(c,d)$) is uniquely defined, and we write $L=Du (c,d)$; moreover, $L$ is given by the partial derivatives of $u$, i.e \[
L (x,y)= \left(\frac{\partial u}{\partial x} (c,d)\right)x + \left( \frac{\partial u}{\partial y}(c,d)\right)y
.\]  

The answer to the above question is \textbf{no} (otherwise complex analysis would be useless!). Now we want to characterise differentiability of $f$ in terms of $u$ and $v$. 

\begin{theorem}[Cauchy-Riemann equations]
     This theorem states that $f=u+iv: \ U \rightarrow \mathbb{C}$ is differentiable at $w=c+id \in U$ if and only if 

     $u,v$ are differentiable at $(c,d) \in U$ \textbf{and} $u,v$ satisfy the Cauchy-Riemann equations at $(c,d)$: \[
     \frac{\partial u}{\partial x}=\frac{\partial v}{\partial y} \quad \text{and} \quad \frac{\partial u}{\partial y}=- \frac{\partial v}{\partial x}
     .\]  
     If $f$ is differentiable at $w=c+id$, then \[
     f' (w)= \frac{\partial u}{\partial x} (c,d)+i \frac{\partial v}{\partial x} (c,d)
     .\]  There are three other such expressions following from the Cauchy-Riemann equations. 
\end{theorem}
\begin{proof}
     $f$ is differentiable at $w$ with derivative $f' (w)=p+iq$:
     \begin{align*}
         \iff & \lim_{z \rightarrow w} \frac{f (z)- f (w)}{z-w}=p+iq\\
         \iff & \lim_{z \rightarrow w} \frac{f (z)- f (w)- (z-w)(p+iq)}{|z-w|}=0\\
     \end{align*}
     Writing $f=u+iv$ and separating real and imaginary parts, the above holds if and only if 
     \begin{align*}
        & \lim_{(x,y) \rightarrow (c,d)} \frac{u (x,y)- u (c,d)-p (x-c)+q (y-d)}{\sqrt{(x-c)^2+ (y-d)^2}}=0\\
        \text{and } & \lim_{(x,y) \rightarrow (c,d)} \frac{v (x,y)- v (c,d)-q (x-c)+p (y-d)}{\sqrt{(x-c)^2+ (y-d)^2}}=0\\
     \end{align*}
     This is precisely the statement that $u$ is differentiable at $(c,d)$ with $Du (c,d)(x,y)=px-qy$, and $v$ is differentiable at $(c,d)$ with $Dv (c,d)(x,y)=qx+py$. 

     So $\iff$ $u,v$ are differentiable at $(c,d)$ and $u_{x}(c,d)=p=v_{y}(c,d)$, $u_{y}(c,d)=-q=-v_{x}(c,d)$, i.e the Cauchy-Riemann equations hold at $(c,d)$. 

     We also get from the above that if $f$ is differentiable at $w$, then $f' (w)=p+iq=u_{x}(c,d)+iv_{x} (c,d)$.  
\end{proof}
\begin{remark}
     $u,v$ satisfying the Cauchy-Riemann equations at a point does \textbf{not} guarantee differentiability of $f$ on its own. We also can proceed in a more simple way if we simply want to show the reverse implication, by writing \[
     f' (w)=\lim_{z \rightarrow w} \frac{f (z)- f (w)}{z-w}=\lim_{h \rightarrow 0} \frac{f (w+h)-f (w)}{h}
     ,\]
     and then choosing $h = t \in \mathbb{R}$ and $h=it$ (since we can choose any direction we want for $h$) in order to get that $u_{x},u_{y},v_{x},v_{y}$ exist and satisfy the Cauchy-Riemann equations.   
\end{remark}
\begin{example*}[Differentiability (?) of conjugation map]
      Let $f (z)=\overline{z} =x-iy$. For this, $u =x, v=-1$, so $u_{x}=1, v_{y}=-1$ and so the C-R equations are not satisfied and $f$ is not differentiable at any point.  
\end{example*}
\begin{corollary}
     Let $f=u+iv: U \rightarrow \mathbb{C}$. If $u,v$ have continuous partial derivatives at $(c,d) \in U$ and satisfy the C-R equations there, then $f$ is differentiable at $w=c+id$.
     
     In particular, if $u,v$ are $\mathbb{C}^{1}$ functions on $U$ (i.e have continuous partial derivatives in $U$) satisfying the C-R equations in $U$, then $f$ is holomorphic in $U$.
\end{corollary}
\begin{proof}
      Continuity of partial derivatives of $u$ implies that $u$ is differentiable, and similarly for $v$ (IB Analysis \& Topology). So the corollary follows from the C-R theorem.
\end{proof}
Complex differentiability is much more restrictive than real differentiability of real and imaginary parts (because of the additional requirement that the Cauchy-Riemann equations must hold). This leads to surprising theorems compared to the real case, for example 
\begin{itemize}
     \item[(a)] If $f: \mathbb{C} \rightarrow \mathbb{C}$ is holomorphic and bounded, then $f$ is constant! (Liouville's theorem). This is false for real functions; for example $\sin (x): \mathbb{R} \rightarrow \mathbb{R}$ . 
     \item[(b)] If $f: U \rightarrow \mathbb{C}$ is holomorphic, then $f$ is infinitely differentiable on $U$.  
\end{itemize}
We will prove these later on. Note that (b) implies that partial derivatives of $u,v$ of all orders exists. So we can differentiate the Cauchy-Riemann equations to get: 
\begin{align*}
     &(u_{x})_x=(v_{y})_x \implies u_{xx}=v_{yx} \text{  and} \\
     &(u_{y})_y=(-v_{x})_y \implies u_{yy}=-v_{xy}.
\end{align*} 
This gives $\nabla^2 u=u_{xx}+u_{yy}=0$ in $U$. Similarly $\nabla^2 v=0$ in $U$.

This means that real and imaginary parts of a holomorphic function are harmonic. This gives a deep connection between harmonic functions and complex analysis; some theorems can be viewed as giving results about harmonic functions. 

Now we need some definitions before the next corollary.
\begin{definition*}
      \begin{enumerate}
           \item A curve is a continuous map $\gamma: [a,b] \rightarrow \mathbb{C}$, where $[a,b] \subset \mathbb{R}$ is a closed interval. We say $\gamma$ is a $C^{1}$ curve if $\gamma'$ exists and is continuous on $[a,b]$. 
           \item An open set $U \subset \mathbb{C}$ is path-connected if for any two points $z,w \in U$, there is a curve $\gamma: [0,1] \rightarrow U$ such that $\gamma (0)=z$ and $\gamma (1)=w$. 
           \item A domain is a non-empty, open, path-connected subset of $\mathbb{C}$.  
      \end{enumerate}
\end{definition*}
\begin{corollary}
     If $U \in C$ is a domain and $f: U \rightarrow \mathbb{C}$ is holomorphic with $f' (z)=0$ for every $z \in U$, then $f$ is constant.
\end{corollary}
\begin{proof}
      Write $f=u+iv$. By the C-R equations, $f'=0 \implies Du=Dv=0$ in $U$. Since $U$ is a domain, this means (IA Analysis and Topology) that $u$ and $v$ are constant, i.e $f$ is constant.
\end{proof}
Now we want to look at some examples of holomorphic functions other than polynomials on $\mathbb{C}$ and rational functions on their domains. We now look at power series, which will give us a wealth of examples. 

\subsection{Power series}
Recall the next theorem from IA Analysis:
\begin{definition*}[Radius of convergence]
      If $(c_{n})_{n=0}^{ \infty}$ is a sequence of complex numbers, then there is a unique number $R \in [0, \infty]$ such that the power series \[
      \sum_{n=0}^{ \infty}c_{n}(z-a)^{n} \qquad z,a \in \mathbb{C}
      .\] 
      converges absolutely if $|z-a|<R$ and diverges if $|z-a|>R$. If $0<r<R$, then the series converges uniformly wrt $z$ on the compact disk $D_{r}=\left\{z \in \mathbb{C}: \ |z-a|<R\right\}$. 

      We call $R$ the radius of convergence of the power series. Note that when $z=R$, we cannot say anything in general about convergence.
\end{definition*}
\begin{theorem}
      Let $\sum_{n=0}^{ \infty}c_{n}(z-a)^{n}$ be a power series with radius of convergence $R>0$. Fix $a \in \mathbb{C}$, and define $f: D (a,R)\rightarrow \mathbb{C}$ by $f (z)= \sum_{n=0}^{ \infty}c_{n}(z-a)^{n}$. Then
      \begin{enumerate}
           \item $f$ is holomorphic on $D (a,R)$. 
           \item The derived series $\sum_{n=1}^{ \infty}nc_{n}(z-a)^{n-1}$ also has radius of convergence $R$, and \[
           f' (z)=\sum_{n=1}^{ \infty}nc_{n}(z-a)^{n-1} \quad \forall z \in D (a,R)
           .\] 
           \item $f$ has derivatives of all orders on $D (a,R)$, and $c_{n}= \frac{f^{(n)}(a)}{n!}$. 
           \item If $f$ vanishes on $D (a, \varepsilon)$ for some $\varepsilon>0$, then $f \equiv 0$ on $D (a,R)$. 
      \end{enumerate}
\end{theorem}
\begin{proof}
      \textbf{Parts (i) and (ii)}: By considering $g (z)=f (z+a)$, we assume WLOG that $a=0$. So $f (z)=\sum_{n=0}^{ \infty}c_{n}z^{n}$ for $z \in D (0,R)$. 

     The derived series $\sum_{n=1}^{ \infty}nc_{n}z^{n-1}$ will have some radius of convergence $R_1 \in [0, \infty]$. Now let $z \in D (0,R)$ be arbitrary. Choose $\rho$ such that $|z|<\rho<R$. Then since $n |\frac{z}{\rho}|^{n-1} \rightarrow 0 $ as $n \rightarrow \infty$, we have \[
     n |c_n||z|^{n-1}=n |c_n||\frac{z}{\rho}|^{n-1}\rho^{n-1} \leq |c_n|p^{n-1}
     .\]  
     for sufficiently large $n$. Since $\sum_{}^{}|c_{n}|\rho^{n}$ converges, it follows that $\sum_{n=1}^{ \infty}nc_{n}z^{n-1}$ converges. Thus $D (0,R)\subset D (0,R_1 )$, i.e $R_1 \geq R$. Since \[
     |c_{n}||z|^{n} \leq n|c_{n}||z|^{n}=|z|\left(|c_{n}||z|^{n-1}\right)
     ,\] 
     if $\sum_{}^{}n |c_{n}||z|^{n-1}$ converges then so does $\sum_{}^{}|c_{n}||z|^{n}$, so $R_1 \leq R$. So $R_1 =R$. 
     
     To prove that $f$ is differentiable with $f' (z)=\sum_{n=1}^{ \infty}nc_{n}(z-a)^{n-1}$, fix $z \in D (0,R)$. The key idea is that this is equivalent to continuity at $z$ of the function 
     \begin{equation*}
          g: D (0,R) \rightarrow \mathbb{C}, \quad g (w)=
           \begin{cases}
                \frac{f (w)- f (z)}{w-z} & w \neq z\\
                \sum_{n=1}^{ \infty}nc_{n}z^{n-1} & w=z.
           \end{cases}          
     \end{equation*}
     By subbing in $f$ we can write $g (w)=\sum_{n=1}^{ \infty}h_{n}(w)$ where 
     \begin{equation*}
          h_{n} (w)=
          \begin{cases}
               \frac{c_{n}(w^{n}-z^{n})}{w-z} & w \neq z\\
               nc_{n}z^{n-1} & w=z.
          \end{cases}  
     \end{equation*}
     Now $h_{n}$ is continuous on $D (0,R)$ (since $w \rightarrow w^{n}$ is differentiable with derivative $nw^{n-1}$). Using $ \frac{w^{n}-z^{n}}{w-z}=\sum_{j=0}^{n-1}z^{j}w^{n-1-j}$, we get that for any $r$ with $|z|<r<R$ and any $w \in D (0,r), |h_{n}(w)|\leq n |c_{n}|r^{n-1} \equiv M_{n}$. Since $\sum M_{n} < \infty$, by the Weierstrass M-test that $\sum_{}^{}h_{n}$ converges uniformly on $D (0,r)$. But a uniform limit of continuous functions is continuous, so $g = \sum_{}^{}h_{n}$ is continuous in $D (0,r)$ and in particular at $z$.

     \textbf{Part (iii)}: Repeatedly apply (ii). The formula $c_{n}= \frac{f^{(n)(a)}}{n!}$ follows by differentiating the series $n$ times and setting $z=a$. 
     
     \textbf{Part (iv)}: If $f=0$ in $D (a, \varepsilon)$, then $f^{(n)}(a)=0$ for all $n$, so $c_{n}=0$ for all $n$ and hence $f=0$ in $D (a,R)$. 
\end{proof}
If $f: \mathbb{C} \rightarrow \mathbb{C}$ is holomorphic on all of $\mathbb{C}$, we say $f$ is entire. 

\begin{proposition}\label{eprop}
      The complex exponential function is defined by \[
      e^{z}= \operatorname{exp}(z)= \sum_{n=0}^{ \infty} \frac{z^{n}}{n!}
      .\] 
      \begin{itemize}
           \item[(i)] $e^{z}$ is entire, with $(e^{z})'=e^{z}$. 
           \item[(ii)] $e^{z} \neq 0$ and $e^{z+w}=e^{z}e^{w}$ for all $z,w \in \mathbb{C}$.
           \item[(iii)] $e^{x+iy}=e^{x}(\cos (y)+i sin (y))$ for $x,y \in \mathbb{R}$. 
           \item[(iv)] $e^{z}=1$ iff $z=2n \pi i$ for some $n \in \mathbb{Z}$. 
           \item[(v)] Let $z \in \mathbb{C}$. There exists $w \in \mathbb{C}$ such that $e^{w}=z$ iff $z \neq 0$.  
      \end{itemize}
\end{proposition}
\begin{proof}
     \textbf{Part (i):} The r.o.c of the series is $ \infty$. To see $(e^{z}=z'),$ differentiate the series term by term using the previous theorem. 

     \textbf{Part (ii):} Fix any $w \in \mathbb{C}$ and set $F (z)= e^{z+w}e^{-z}$. Then $F' (z)=-e^{z+w}e^{-z}+e^{z+w}e^{-z}=0$, so $F(z)$ is a constant. Thus $F (z)=F (0)=e^{w}$ for all $z \in \mathbb{C}$. Thus \[
     e^{z+w}e^{-z}=e^{w} \quad \forall z,w \in \mathbb{C}
     .\] Taking $w=0$, $e^{z}e^{-z}=1$. So $e^{z} \neq 0$. Multiplying by $e^{z}$, we get $e^{z+w}=e^{z}e^{w}$. 

     \textbf{Part (iii):} $e^{x+iy}=e^{x}e^{iy}$ by (ii). Now use the definition of $e^{iy}$, and the series for $\sin (y), \cos (y)$ for $y \in \mathbb{R}$.

     \textbf{Part (iv) and (v):} Follow from (iii). (Exercise)
\end{proof}
\begin{definition*}[Logarithm]
      Given $z \in \mathbb{C}$, we say a complex $w \in \mathbb{C}$ is a \vocab{logarithm} of $z$ if $e^{w}=z$. 

      By Proposition \ref{eprop}(v), $z$ has a logarithm iff $z \neq 0$. By (ii) and (iv), if $z \neq 0$ then $z$ has infinitely many logarithms, with any two differing from each other by $2n \pi i$ for some integer $n$. 
      
      If $w$ is a logarithm of $z$, then $e^{\operatorname{Re}(w)}=|z|$, so $\operatorname{Re}(w)=\operatorname{log} |z|$ (the real logarithm of the positive number $|z|$); in particular, this is well-defined.
\end{definition*}
\begin{definition*}[Branch of a logarithm]
      Let $U \subset \mathbb{C} \backslash \left\{0\right\}$ be open. Then a branch of logarithm on $U$ is a continuous function $\lambda: U \rightarrow \mathbb{C}$ such that $e^{\lambda (z)}=z$ for each $z \in U$.
\end{definition*}
\begin{proposition}
      If $\lambda$ is a branch of log on $U$ then $\lambda$ is automatically holomorphic in $U$, with $\lambda' (z)=\frac{1}{z}$.
\end{proposition}
\begin{proof}
      If $w \in U$ then 
      \begin{align*}
           \lim_{z \rightarrow w} \frac{\lambda (z)- \lambda (w)}{z-w}&=\lim_{z \rightarrow w} \frac{1}{\left( \frac{e^{\lambda (z)}-e^{\lambda (w)}}{\lambda (z)- \lambda (w)}\right)}\\
           &=\frac{1}{e^{\lambda (w)}}\lim_{z \rightarrow w} \frac{1}{\left(\frac{e^{\lambda (z)-\lambda (w)}-1}{\lambda (z)- \lambda (w)}\right)}\\
           &=\frac{1}{e^{\lambda (w)}} \lim_{h \rightarrow 0} \frac{1}{\left( \frac{e^{h}-1}{h}\right)} \quad \text{ since } \lambda \text{ is continuous } \\
           &=\frac{1}{e^{\lambda (w)}}=\frac{1}{w}.
      \end{align*}
\end{proof}
\begin{definition*}[Principal branch of logarithm]
      The principal branch of logarithm is the function \[
      \operatorname{Log}: U_1 =\mathbb{C} \backslash \left\{x \in \mathbb{R}: \ x \leq 0\right\} \rightarrow \mathbb{C}
      .\] defined by \[
      \operatorname{Log}(z)= \operatorname{log}|z|+i \operatorname{arg}(z)
      .\] where $\operatorname{arg}(z)$ is the unique argument of $z \in U_1 $ in $(-\pi,\pi)$.
\end{definition*}
\begin{remark}
     $\operatorname{Log}$ is a branch of logarithm in $U_{1}:$ to check continuity of $\operatorname{Log}$, note that
     $z \mapsto \log |z|$ is continuous on $\mathbb{C} \backslash\{0\}$ (by continuity of $z \mapsto|z|$ and
     continuity of $r \mapsto \log r$ for $r>0$ ); also, $z \mapsto \arg (z)$ is continuous, since
     $\theta \mapsto e^{i \theta}$ is a homeomorphism $(-\pi, \pi) \rightarrow \mathbb{S}^{1} \backslash\{-1\}$ (as can be checked
     directly, where $\left.\mathbb{S}^{1}=\{z:|z|=1\}\right)$, and $z \mapsto \frac{z}{|z|}$ is continuous on
     $\mathbb{C} \backslash\{0\} .$ So $z \mapsto \log (z)$ is continuous on $U_{1}$.
     We also have \[
          e^{\operatorname{Log} (z)}=e^{\ln |z|+i \arg (z)}=e^{\ln |z|} \cdot e^{i \text { arg }(z)}=
          |z|(\cos \arg (z)+i \sin \arg (z))=z
     .\]  So $\operatorname{Log}$ is a branch of logarithm in $U_{1}$.
\end{remark}
\begin{remark}
      $\operatorname{Log}$ does not have a continuous extension to $\mathbb{C} \backslash\{0\}$ since $\arg (z) \rightarrow \pi$ as $z \rightarrow-1$ with $\operatorname{Im}(z)>0$, and $\arg (z) \rightarrow-\pi$ as $z \rightarrow-1$ with $\operatorname{Im}(z)<0$.
\end{remark}
\begin{proposition}
      \begin{itemize}
           \item[(i)] Log is holomorphic on $U_1 $ with $\operatorname{Log}' (z)=\frac{1}{z}$. 
           \item[(ii)] For $|z|<1$, we have \[
           \operatorname{Log}(1+z)=\sum_{n=1}^{ \infty} \frac{(-1)^{n-1}z^{n}}{n}
           .\] 
      \end{itemize} 
\end{proposition}
\begin{proof}
      \begin{itemize}
           \item[(i)] is the remark above. 
           \item[(ii)] To see this, note that the R.O.C of the series is 1, and $|z|<1 \implies 1+z \in U_1 $, so both sides are defined on $|z|<1$. 
           
           Let $F (z)=\operatorname{Log}(1+z)-\sum_{n=1}^{ \infty} \frac{(-1)^{n-1}z^{n}}{n}$ for $|z|<1$. Then \[
           F' (z)= \frac{1}{1+z}-\sum_{n=1}^{ \infty}(-z)^{n-1}=0
           .\] So $F (z)=\text{ constant } =F (0)=0$.
      \end{itemize}
\end{proof}
Using exp and Log we can define further useful functions. 
\begin{enumerate}
     \item For any $\alpha \in \mathbb{C}$, define \[
     z^{\alpha}=e^{\alpha \operatorname{Log}(z)}, \quad z \in U_1 
     .\] This is the principal branch of $z^{\alpha}$. It's holomorphic on $U_1 $ with derivative $\alpha z^{\alpha-1}$. 
     \item We can define the familiar functions 
     \begin{itemize}
          \item $\cos (z)= \frac{e^{iz}+e^{-iz}}{2}$ 
          \item $\sin (z)= \frac{e^{iz}-e^{-iz}}{2i}$ 
          \item $\cosh (z)= \frac{e^{z}+e^{-z}}{2}$ 
          \item $\sinh (z)= \frac{e^{z}-e^{-z}}{2}$ 
     \end{itemize}
     These are all entire since exp is entire, with derivatives given by the familiar expressions from real variables.
\end{enumerate}
\subsection{Conformality}
Let $f: U \rightarrow \mathbb{C}$ be holomorphic ($U \subset C $ is open). Let $w \in U$ and suppose that $f' (w) \neq 0$. Take two $C^{1}$ curves $\gamma_1 , \gamma_2 : [-1,1] \rightarrow U$ such that $\gamma_1 (0)=\gamma_2 (0)=w$ and with nonzero derivative. Then $f \circ \gamma_i$ are $C^{1}$ curves passing through $f (w)$. Moreover, $(f \circ \gamma_i)' (0)=f' (w)\gamma_{i}' (0)\neq 0$. Thus \[
\frac{(f \circ \gamma_1 )' (0)}{(f \circ \gamma_2 )' (0)}= \frac{\gamma_1 ' (0)}{\gamma_2 ' (0)}
.\]
Hence \[
\operatorname{arg}(f \circ \gamma_1 )' (0)-\operatorname{arg}(f \circ \gamma_2 )' (0)=\operatorname{arg}\gamma_1 ' (0)-\operatorname{arg}\gamma_2 ' (0)
.\] This means that the angle that the curves $\gamma_1 , \gamma_2 $ make at $w$ is the same as the angle their images make at $f (w)$. We say f is `angle-preserving at $w$', whenever $f' (w) \neq 0$.
\begin{remark}
      If $f$ is a $C^{1}$ map on $U$, the converse of this also holds. See Example Sheet 1. 
\end{remark}
\begin{definition*}[Conformal map]
      A holomorphic function $f: U \rightarrow \mathbb{C}$ on an open set $U$ is said to be \vocab{conformal} at a point $w \in U$ if $f' (w) \neq 0$.
\end{definition*}
\begin{definition*}[Conformal equivalence]
      Let $U, \widetilde{U} $ be domains in $\mathbb{C}$. A map $f: U \rightarrow \widetilde{U}$ is said to be a conformal equivalence between $U$ and $\widetilde{U}$ if f is a bijective holomorphic map with $f' (z) \neq 0$ for every $z \in U$.
\end{definition*}
\begin{remark}
      \begin{itemize}
           \item If $f$ is holomorphic and injective, then $f' (z) \neq 0$ for each $z$. We will prove this later. So in the above definition $f' (z) \neq 0$ is redundant. 
           \item It is automatic that the inverse ${f}^{-1}: U \rightarrow \widetilde{U}$ is holomorphic. This can be proved using the holomorphic inverse function theorem, which you will prove on Example Sheet 1.
      \end{itemize}
\end{remark}
\begin{example*}
     Let's look at some examples of conformal equivalence.
      \begin{enumerate}
           \item Mobius maps are defined for $a,b,c,d \in \mathbb{C}$ with $ad-bc \neq 0$ (see IA Groups): \[
           f (z)= \frac{az+b}{cz+d}
           .\] Mobius maps sometimes serve as explicit conformal equivalences between subdomains of $\mathbb{C}$. For example, let $\mathbb{H}$ be the open upper half plane. Then 
           \begin{align*}
                z \in \mathbb{H} &\iff |z-i| < |z+i|\\
                & \iff | \frac{z-i}{z+i}|<1.
           \end{align*}
           Thus $g (z)=\frac{z-i}{z+i}$ maps $\mathbb{H}$ onto $D (0,1)$, so $g$ is a conformal equivalence.
           \item Consider $f: z \rightarrow z^{n}$ where $n \in \mathbb{N}$ where $f: \left\{z \in C \backslash \left\{0\right\}: 0 < \operatorname{arg}(z) < \frac{\pi}{n}\right\} \rightarrow \mathbb{H}$. This is a conformal equivalence with inverse $f (z)=z^{1/n}$ (the principal branch).
           \item We have that \[
           \operatorname{exp}: \left\{z \in \mathbb{C}: \ -\pi < \operatorname{Im}(z)<\pi\right\} \rightarrow \mathbb{C} \backslash \left\{x \in \mathbb{R}: x \leq 0\right\}
           .\]  
      \end{enumerate}
\end{example*}
Aside:
\begin{theorem}[Riemann Mapping Theorem]
      Any simply connected domain $U \subset \mathbb{C}$ with $U \neq \mathbb{C}$ is conformally equivalent to $D (0,1)$.
\end{theorem}
\begin{proof}
      This is beyond the scope of the course.\footnotemark
\end{proof}\footnotetext{See Rudin's \emph{Real and Complex Analysis}.}
\section{Complex Integration: Part I}
We aim to extend Riemann integration to complex functions $f: U \rightarrow \mathbb{C}$ along curves in $U$. First we take a look at complex functions of a real variable. 
\begin{definition*}
      If $f: [a,b] \subset \mathbb{R} \rightarrow C$ is a complex function and if $f$ is Riemann integrable, define \[
      \int_{a}^{b}f (t) \mathrm{d}t= \int_{a}^{b}\operatorname{Re} f (t) \mathrm{d}t+ \int_{a}^{b}\operatorname{Im}f (t) \mathrm{d}t   
      .\] In particular, \[
      \int_{a}^{b}i f (t) \mathrm{d}t =i \int_{a}^{b}f (t) \mathrm{d}t  
      .\] We can then directly calculate for any $w \in \mathbb{C}$ that \[
          \int_{a}^{b}w f (t) \mathrm{d}t =w\int_{a}^{b}f (t) \mathrm{d}t 
      .\] 
\end{definition*}
\begin{proposition}
      If $f: [a,b] \rightarrow \mathbb{C}$ is continuous, then \[
      |\int_{a}^{b} f (t) \mathrm{d}t | \leq \int_{a}^{b} |f (t)| \mathrm{d}t \leq (b-a) \sup_{t \in [a,b]} |f (t)|
      ,\] with equality iff $f$ is constant. 
\end{proposition}
\begin{proof}
      If $\int_{a}^{b}f (t) \mathrm{d}t $ then we are done. Else write $\int_{a}^{b}f (t) \mathrm{d}t= re^{i \theta} $ for some $\theta \in [0,2\pi)$ and let $M=\sup_{t \in [a,b]} |f (t)|$. Then 
      \begin{align*}
           |\int_{a}^{b}f (t) \mathrm{d}t| &=r= e^{-i \theta}\int_{a}^{b}f (t) \mathrm{d}t=\int_{a}^{b}e^{-i \theta}f (t) \mathrm{d}t\\
           &=\int_{a}^{b}\operatorname{Re}(e^{-i \theta}) f (t) \mathrm{d}t+ \int_{a}^{b} \operatorname{Im}(e^{-i \theta})f (t) \mathrm{d}t. 
      \end{align*}
      Since the LHS is real, \[
      |\int_{a}^{b}f (t) \mathrm{d}t|=\int_{a}^{b}\operatorname{Re} f (t) \mathrm{d}t \leq  \int_{a}^{b}|e^{-i \theta} f (t)| \mathrm{d}t=\int_{a}^{b}| f (t)| \mathrm{d}t\leq (b-a)M
      .\] 
      Equality holds iff $|f (t)|=M$ and $\operatorname{Re} (e ^{-i \theta}f (t))=M$ for all $t \in [a,b]$, i.e iff $|f (t)|=M$ and $\operatorname{arg} f (t)= \theta$ for all $t$; iff $f$ constant.
\end{proof}
\begin{definition*}[Integral along a curve]
      Let $U \subset \mathbb{C}$ be open and $f: U \rightarrow \mathbb{C}$ be continuous. Let $\gamma: [a,b] \rightarrow U$ be a $C^1$ curve. Then the \vocab{integral of $f$ along $\gamma$} is \[
      \int_{\gamma}^{}f (z) \mathrm{d}z= \int_{a}^{b} f (\gamma (t))\gamma' (t) \mathrm{d}t  
      .\]
\end{definition*}
\begin{proposition}[Basic properties of the integral]
      If we have the integral of $f$ along $\gamma$, we have the following properties: 
      \begin{enumerate}
           \item Invariance under reparametrisation: Let $\varphi: [a_1 ,b_1 ] \rightarrow [a,b]$ be $C^1$ and injective with $\varphi (a_1 )=a, \varphi (b_1 )=b.$ Let $\delta= \gamma \circ \varphi : [a_1 ,b_1] \rightarrow U$. Then we have \[
           \int_{\delta}^{}f (z) \mathrm{d}z= \int_{\gamma}^{}f (z) \mathrm{d}z  
           .\]  
           \item Linearity: \[
           \int_{\gamma}^{}c_1 f_1 (z)+ c_2 f_2 (z) \mathrm{d}z= c_1 \int_{\gamma}^{}f_1 (z) \mathrm{d}z +c_2  \int_{\gamma}^{}f_2 (z) \mathrm{d}z 
           .\] 
           \item Additivity: If $\gamma$ is our $C^1$ curve and $a<c<b$, then \[
           \int_{\gamma}^{}f (z) \mathrm{d}z = \int_{\gamma|_[a,c]}^{}f (z) \mathrm{d}z + \int_{\gamma|_[c,b]}^{}f (z) \mathrm{d}z  
           .\] 
           \item Inverse path: Define the inverse path $(-\gamma):[-b,-a] \rightarrow U$ by $(-\gamma)(t)=\gamma(-t)$ for $-b \leq t \leq-a$. Then
           $$
           \int_{(-\gamma)} f(z) d z=-\int_{\gamma} f(z) \mathrm{d} z
           $$
      \end{enumerate}
\end{proposition}
\begin{proof}
     For 1, we have 
     \begin{align*}
          \int_{\delta}^{}f (z) \mathrm{d}z &= \int_{a_1 }^{b_1 }f (\upsilon \circ \varphi (t)) \gamma' (\varphi (t))\varphi' (t) \mathrm{d}t \\
          &=\int_{a}^{b}f (\gamma (s))\gamma' (s) \mathrm{d}s = \int_{\gamma}^{}f (z) \mathrm{d}z \quad \text{ by change of vars. } s=\varphi (t).
      \end{align*}
      2,3, and 4 are all easy to check from the definition.
\end{proof}
\begin{definition*}[Length of a curve]
     Definition: Let $\gamma:[a, b] \rightarrow \mathbb{C}$ be a $C^{1}$ curve. The length of $\gamma$ is defined by
     $$
     \operatorname{length}(\gamma)=\int_{a}^{b}\left|\gamma^{\prime}(t)\right| d t.
     $$
\end{definition*}
\begin{definition*}
      A \vocab{piecewise $C^1$ curve} is a continuous map $\gamma: [a,b] \rightarrow \mathbb{C}$ such that there exists a finite subdivision \[
      a=a_0 <a_1 < \ldots <a_{n-1}<a_{n}=b
      \] with the property that $\gamma_{j}=\gamma|_{[a_{j-1},a_j]}: \ [a_{j-1},a_j] \rightarrow \mathbb{C}$ is $C^{1}$ for all $1 \leq j \leq n$. We define \[
      \int_{\gamma}^{}f (z) \ \mathrm{d} z= \sum_{j=1}^{n} \int_{\gamma_{j}}^{}f (z) \ \mathrm{d}z 
      .\] 
      and \[
          \operatorname{length}(\gamma)=\sum_{j=1}^{n}\operatorname{length}(\gamma_{j})=\sum_{j=1}^{n}\int_{a_{j-1}}^{a_j}|\gamma' (t)| \ \mathrm{d}t 
      .\] 
\end{definition*}
\begin{remark}
      From now on, by a `curve' we shall mean a piecewise $C^{1}$ curve.
\end{remark}
\begin{definition*}
     If $\gamma_{1}:[a, b] \rightarrow \mathbb{C}$ and $\gamma_{2}:[c, d] \rightarrow \mathbb{C}$ are curves with $\gamma_{1}(b)=\gamma_{2}(c)$, we define the sum of of $\gamma_{1}$ and $\gamma_{2}$ to be the curve \[
          \left(\gamma_{1}+\gamma_{2}\right): [a, b+d-c] \rightarrow \mathbb{C}
     ,\] 
     \begin{equation*}
          \left(\gamma_{1}+\gamma_{2}\right)(t)=
           \begin{cases}
               \gamma_{1}(t) & a \leq t \leq b \\
               \gamma_{2}(t-b+c) & b \leq t \leq b+d-c
          \end{cases}
     \end{equation*}
\end{definition*}
\begin{proposition}
     For any continuous function $f: U \rightarrow \mathbb{C}$ and any curve $\gamma:[a, b] \rightarrow \mathbb{C}$, we have that
     $$
     \left|\int_{\gamma} f(z) d z\right| \leq \text { length }(\gamma) \sup _{\gamma}|f|
     $$
     where $\sup _{\gamma}|f|=\sup _{t \in[a, b]}|f(\gamma(t))|$.
\end{proposition}
\begin{proof}
     If $\gamma$ is $C^{1}$, then \[
          \left|\int_{\gamma} f(z) d z\right|=\left|\int_{a}^{b} f(\gamma(t)) \gamma^{\prime}(t) d t\right| \leq
          \int_{a}^{b}\left|f(\gamma(t)) \| \gamma^{\prime}(t)\right| d t \leq \sup _{t \in[a, b]}|f(\gamma(t))|\operatorname{length}(\gamma)
     .\] If $\gamma$ is piecewise $C^{1}$ then the result follows from the definition $\int_{\gamma} f(z) d z=\sum_{j=1}^{n} \int_{\gamma_{j}} f(z) d z$ where $\gamma_{j}$ is $C^{1}$, and the triangle inequality.
\end{proof}
We can now look at the complex version of the FTC.
\begin{theorem}[Fundamental Theorem of Calculus]
     Suppose that $f: U \rightarrow \mathbb{C}$ is continuous, $U \subset \mathbb{C}$ open. If there is a function $F: U \rightarrow \mathbb{C}$ such that $F^{\prime}(z)=f(z)$ for all $z \in U$, then for any curve $\gamma:[a, b] \rightarrow U$,
     $$
     \int_{\gamma} f(z) d z=F(\gamma(b))-F(\gamma(a)) .
     $$
     If additionally $\gamma$ is a closed curve, i.e. $\gamma(b)=\gamma(a)$, then $\int_{\gamma} f(z) d z=0$.
\end{theorem}
\begin{proof}
     This follows immediately:
      \[
          \int_{\gamma} f(z) d z=\int_{a}^{b} f(\gamma(t)) \gamma^{\prime}(t) d t=\int_{a}^{b} \frac{d}{d t} F(\gamma(t)) d t=F(\gamma(b))-F(\gamma(a))
      .\] 
\end{proof}
\begin{remark}
     Such $F$ as in Theorem $2.3$ is called an \vocab{anti-derivative} of $f$.
     
     We shall see later (by infinite differentiability of holomorphic functions) that if $F^{\prime}(z)=f(z)$, then $f$ is automatically continuous.
\end{remark}
\begin{example*}
      Let $\int_{\gamma}^{}z^{n} \ \mathrm{d}z $ for $n \in \mathbb{Z}$, where $\gamma: [0,1]\rightarrow \mathbb{C}$, $\gamma (t)=R e^{2\pi it}$ for some $R>0$. (The image of $\gamma$ is the circle of radius $R$ centred at 0). 

      For $n \neq 1$, $ \frac{z^{n+1}}{n+1}$ is an antiderivative of $z^{n}$ in $\mathbb{C}\backslash \left\{0\right\}$, so by the FTC, $\int_{\gamma}^{}z^{n} \ \mathrm{d}z =0$ since $\gamma$ is a closed curve.

      For $n=-1$, use the definition of the integral: \[
          \int_{\gamma}^{}\frac{1}{z} \ \mathrm{d}z =\int_{0}^{1} \frac{\gamma' (t)}{\gamma (t)} \ \mathrm{d}t=\int_{0}^{1} \frac{2 \pi iR e^{2\pi it}}{R e^{2\pi it}} \ \mathrm{d}t= 2 \pi i
      .\] 
      Since $\int_{\gamma} \frac{1}{z} d z \neq 0$, we can conclude that for any $R>0, \frac{1}{z}$ has no anti-derivative in any open set containing the circle $\{|z|=R\}$.

      In particular, since for any branch $\lambda(z)$ of logarithm the derivative $\lambda^{\prime}(z)=\frac{1}{z}$, there is no branch of logarithm on $\mathbb{C}^{\star}=\mathbb{C} \backslash\{0\} .$
\end{example*}
\begin{theorem}[Converse to FTC]
      
\end{theorem}
\end{document}